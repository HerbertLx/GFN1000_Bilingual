\documentclass{article}
\usepackage[utf8]{inputenc} % 支持UTF-8编码
\usepackage{xeCJK} % 支持中文
\usepackage{graphicx} % 引入用于插入图片的宏包
\usepackage{hyperref} % 引入超链接宏包
\usepackage{amsmath}

\usepackage{geometry}
\geometry{
  a4paper,
  left=20mm,
  right=20mm,
  top=30mm,
  bottom=30mm
}

% 设置行距为1.5倍
\usepackage{setspace}
\linespread{1.75}

\begin{document}

\title{\textbf{
GFN1000 与自然对话\\
中英双语对照翻译版
}} % 文章标题
\date{}
\maketitle % 生成标题

\setcounter{secnumdepth}{0} % 禁止章节编号,但仍添加到目录
\tableofcontents
\newpage
\section{Text 7 from Science and Method/ \textit{Henri Poincaré}}
\begin{center}
I. THE SELECTION OF FACTS 事实的选择 
\end{center}

\noindent 1\\
Tolstoi explains somewhere in his writings why, in his opinion, “Science for Science’s sake” is an absurd conception. We cannot know all the facts, since they are practically infinite in number. We must make a selection; and that being so, can this selection be governed by the mere caprice of our curiosity? Is it not better to be guided by utility, by our practical, and more especially our moral, necessities? Have we not some better occupation than counting the number of lady-birds in existence on this planet?\\
托尔斯泰在他的著作中的某个地方解释了,在他看来,为什么“为了科学而科学” 是一个荒谬的概念。我们不可能知晓所有的事实,因为事实上事实的数量是无穷无尽的。我们必须做出选择;既然如此,这种选择难道应该仅仅由我们好奇心的一时兴起所决定吗?难道不应该以实用性,以我们实际的、尤其是道德上的需求为导向吗?难道我们就没有比数清这个星球上现有瓢虫数量更好的事情可做吗? \\

\noindent 2\\
It is clear that for him the word *utility* has not the meaning assigned to it by business men, and, after them, by the greater number of our contemporaries. He cares but little for the industrial applications of science, for the marvels of electricity or of automobilism, which he regards rather as hindrances to moral progress. For him the useful is exclusively what is capable of making men better.\\
显然,对他(托尔斯泰)来说,“实用性”(utility)这个词并不具有商人们赋予它的含义,而在商人之后,我们同时代的大多数人也沿用了商人对这个词的定义。他对科学的工业应用,对电力或汽车领域的种种神奇发明并不太在意,相反,他认为这些东西是道德进步的阻碍。对他而言,真正有用的东西,唯有那些能够使人变得更好的事物。 \\ 

\noindent 3\\
It is hardly necessary for me to state that, for my part, I could not be satisfied with either of these ideals. I have no liking either for a greedy and narrow plutocracy, or for a virtuous unaspiring democracy, solely occupied in turning the other cheek, in which we should find good people devoid of curiosity, who, avoiding all excesses, would not die of any disease—save boredom. But it is all a matter of taste, and that is not the point I wish to discuss.\\
几乎无需我多说,就我个人而言,这两种理念(观点)我都无法认同。我既不喜欢贪婪狭隘的财阀统治,也不喜欢那种只知逆来顺受、毫无抱负的 “道德” 民主制。在这样的民主制度下,人们虽善良却缺乏好奇心,他们避免一切极端行为,除了会被无聊 “折磨” 外,不会因任何疾病而死。但这完全是个人喜好问题,并非我想讨论的重点。 \\ 

\noindent 4\\
Nonetheless the question remains, and it claims our attention. If our selection is only determined by caprice or by immediate necessity, there can be no science for science’s sake, and consequently no science. Is this true? There is no disputing the fact that a selection must be made: however great our activity, facts outstrip us, and we can never overtake them; while the scientist is discovering one fact, millions and millions are produced in every cubic inch of his body. Trying to make science contain nature is like trying to make the part contain the whole.\\
尽管如此,问题依然存在,并且值得我们关注。如果我们对事实的选择仅仅取决于一时的心血来潮或当下的需求,那就不会有为了科学而进行的科学研究,进而也就没有科学可言。这是真的吗?毫无疑问,我们必须做出选择:无论我们多么积极探索,事实的数量都远超我们所能掌握的,我们永远也无法穷尽它们;当科学家发现一个事实时,他身体的每立方英寸内都在产生数以百万计的新事实。试图让科学涵盖自然万物,就如同试图让部分包含整体一样。\\ 

\noindent 5\\
But scientists believe that there is a hierarchy of facts, and that a judicious selection can be made. They are right, for otherwise there would be no science, and science does exist. One has only to open one’s eyes to see that the triumphs of industry, which have enriched so many practical men, would never have seen the light if only these practical men had existed, and if they had not been preceded by disinterested fools who died poor, who never thought of the useful, and yet had a guide that was not their own caprice.\\
但是科学家们相信,事实存在一个等级体系,并且可以做出明智的选择。他们是对的,否则就不会有科学,而科学确实是存在的。只要稍加留意就会发现,如果只有那些注重实际的人存在,而没有之前那些无私奉献、穷困潦倒,从不考虑实用性,但又有着并非出于心血来潮的指引的 “傻瓜”,那么那些让众多务实之人致富的工业成就,就永远不会出现。 \\ 

\noindent 6\\
What these fools did, as Mach has said, was to save their successors the trouble of thinking. If they had worked solely in view of an immediate application, they would have left nothing behind them, and in face of a new requirement, all would have had to be done again. Now the majority of men do not like thinking, and this is perhaps a good thing, since instinct guides them, and very often better than reason would guide a pure intelligence, at least whenever they are pursuing an end that is immediate and always the same. But instinct is routine, and if it were not fertilized by thought, it would advance no further with man than with the bee or the ant. It is necessary, therefore, to think for those who do not like thinking, and as they are many, each one of our thoughts must be useful in as many circumstances as possible. For this reason, the more general a law is, the greater is its value.\\
正如马赫所说,这些 “傻瓜” 所做的是为他们的后继者省去思考的麻烦。如果他们仅仅为了即刻的应用而工作,那他们将不会留下任何成果,面对新的需求时,一切都得重新开始。如今,大多数人不喜欢思考,这或许是件好事,因为本能引导着他们,而且在追求即时且一成不变的目标时,本能常常比理性对纯粹智力的引导更有效。但本能是循规蹈矩的,如果没有思想的 “孕育”,人类的进步不会比蜜蜂或蚂蚁更多。因此,有必要为那些不喜欢思考的人去思考,而这样的人很多,所以我们的每一个思想都必须尽可能在多种情况下有用。基于这个原因,一条定律越具普遍性,它的价值就越大。 \\

\noindent 7\\
This shows us how our selection should be made. The most interesting facts are those which can be used several times, those which have a chance of recurring. We have been fortunate enough to be born in a world where there are such facts. Suppose that instead of eighty chemical elements we had eighty millions, and that they were not some common and others rare, but uniformly distributed. Then each time we picked up a new pebble there would be a strong probability that it was composed of some unknown substance. Nothing that we knew of other pebbles would tell us anything about it. Before each new object we should be like a new - born child; like him we could but obey our caprices or our necessities. In such a world there would be no science, perhaps thought and even life would be impossible, since evolution could not have developed the instincts of self - preservation. Providentially it is not so; but this blessing, like all those to which we are accustomed, is not appreciated at its true value. The biologist would be equally embarrassed if there were only individuals and no species, and if heredity did not make children resemble their parents.\\
这告诉我们应该如何做出选择。最有趣的事实是那些可以被多次运用、有可能反复出现的事实。我们很幸运,出生在一个存在这类事实的世界里。假设世界上不是80种化学元素,而是8000万种,并且它们并非有常见和稀有之分,而是均匀分布的。那么每次我们捡起一块新鹅卵石时,它很有可能是由某种未知物质构成的。我们对其他鹅卵石的了解,并不能让我们对这块新的鹅卵石有任何认知。面对每一个新事物,我们就会像新生儿一样;像新生儿一样,我们只能凭一时的兴致或实际需求行事。在这样一个世界里,科学将不复存在,也许思想乃至生命都不可能存在,因为进化无法发展出自我保护的本能。幸运的是,事实并非如此;但这份幸运,就像我们习以为常的所有事物一样,其真正价值并未得到珍视。如果只有个体而没有物种,并且遗传无法让子代与亲代相似,生物学家也会同样陷入困境。 \\

\noindent 8\\
Which, then, are the facts that have a chance of recurring? In the first place, simple facts. It is evident that in a complex fact many circumstances are united by chance, and that only a still more improbable chance could ever so unite them again. But are there such things as simple facts? and if there are, how are we to recognize them? Who can tell that what we believe to be simple does not conceal an alarming complexity? All that we can say is that we must prefer facts which appear simple, to those in which our rude vision detects dissimilar elements. Then only two alternatives are possible; either this simplicity is real, or else the elements are so intimately mingled that they do not admit of being distinguished. In the first case we have a chance of meeting the same simple fact again, either in all its purity, or itself entering as an element into some complex whole. In the second case the intimate mixture has similarly a greater chance of being reproduced than a heterogeneous assemblage. Chance can mingle, but it cannot unmingle, and a combination of various elements in a well-ordered edifice in which something can be distinguished, can only be made deliberately. There is, therefore, but little chance that an assemblage in which different things can be distinguished should ever be reproduced. On the other hand, there is great probability that a mixture which appears homogeneous at first sight will be reproduced several times. Accordingly facts which appear simple, even if they are not so in reality, will be more easily brought about again by chance.\\
那么,哪些事实有可能再次出现呢?首先是简单的事实。显然,在一个复杂的事实中,许多情况是偶然结合在一起的,而只有更罕见的偶然才会让它们再次以同样的方式结合。但真的存在简单事实这种东西吗?如果有,我们又该如何识别它们呢?谁能说我们认为是简单的东西,背后没有隐藏着惊人的复杂性呢?我们所能说的是,相比于那些在我们粗略观察下能发现不同元素的事实,我们必须更倾向于看似简单的事实。那么,只有两种可能:要么这种简单性是真实的,要么这些元素混合得非常紧密,以至于无法区分。在第一种情况下,我们有可能再次遇到同样的简单事实,要么是纯粹的形式,要么是作为一个元素融入到某个复杂的整体中。在第二种情况下,紧密的混合比异质的组合更有可能再次出现。偶然可以使事物混合,但不能使它们分离,而且在一个井然有序、其中元素可被区分的 “结构” 中,各种元素的组合只能是刻意为之的。因此,一个能区分出不同事物的组合再次出现的可能性很小。另一方面,乍一看似乎是均匀的混合物则很有可能多次出现。因此,看似简单的事实,即使实际上并非如此,也更有可能因偶然因素再次发生。 \\

\noindent 9\\
It is this that justifies the method instinctively adopted by scientists, and what perhaps justifies it still better is that facts which occur frequently appear to us simple just because we are accustomed to them.\\
正是这一点证明了科学家们本能采用的方法是合理的,或许更能证明其合理性的是,那些频繁出现的事实在我们看来很简单,仅仅是因为我们对它们习以为常。 \\

\noindent 10\\
But where is the simple fact? Scientists have tried to find it in the two extremes, in the infinitely great and in the infinitely small. The astronomer has found it because the distances of the stars are immense, so great that each of them appears only as a point and qualitative differences disappear, and because a point is simpler than a body which has shape and qualities. The physicist, on the other hand, has sought the elementary phenomenon in an imaginary division of bodies into infinitely small atoms, because the conditions of the problem, which undergo slow and continuous variations as we pass from one point of the body to another, may be regarded as constant within each of these little atoms. Similarly the biologist has been led instinctively to regard the cell as more interesting than the whole animal, and the event has proved him right, since cells belonging to the most diverse organisms have greater resemblances, for those who can recognize them, than the organisms themselves. The sociologist is in a more embarrassing position. The elements, which for him are men, are too dissimilar, too variable, too capricious, in a word, too complex themselves. Furthermore, history does not repeat itself; how, then, is he to select the interesting fact, the fact which is repeated? Method is precisely the selection of facts, and accordingly our first care must be to devise a method. Many have been devised because none holds the field undisputed. Nearly every sociological thesis proposes a new method, which, however, its author is very careful not to apply, so that sociology is the science with the greatest number of methods and the least results.\\
但是简单的事实在哪里呢?科学家们试图在两个极端中找到它,即在无穷大和无穷小之中。天文学家找到了简单事实,因为恒星之间的距离极其遥远,远到每颗恒星看起来都只是一个点,质量上的差异也消失了,而且一个点比一个有形状和质量的物体更简单。另一方面,物理学家在将物体假想地分割成无穷小的原子的过程中寻找基本现象,因为当我们从物体的一个点移动到另一个点时,问题的条件会发生缓慢而连续的变化,但在每个这样的小原子内部,这些条件可以被视为是恒定的。同样,生物学家本能地认为细胞比整个动物更有趣,事实也证明他是对的,因为对于能够识别它们的人来说,属于最不同生物体的细胞之间的相似之处,比生物体本身之间的相似之处更多。社会学家的处境则更为尴尬。对他们来说,研究对象是人类,人类彼此之间差异太大、变化太多、反复无常,总之,人类本身就过于复杂。此外,历史不会重演;那么,社会学家该如何选择有趣的、会重复出现的事实呢?方法恰恰就是对事实的选择,因此我们首先要做的就是设计一种方法。人们已经设计出了许多方法,因为没有一种方法能在该领域毫无争议地占据主导地位。几乎每一篇社会学论文都会提出一种新方法,然而,论文作者却小心翼翼地不将其应用,以至于社会学成了方法最多但成果最少的学科。\\ 

\noindent 11\\
It is with regular facts, therefore, that we ought to begin; but as soon as the rule is well established, as soon as it is no longer in doubt, the facts which are in complete conformity with it lose their interest, since they can teach us nothing new. Then it is the exception which becomes important. We cease to look for resemblances, and apply ourselves before all else to differences, and of these differences we select first those that are most accentuated, not only because they are the most striking, but because they will be the most instructive. This will be best explained by a simple example. Suppose we are seeking to determine a curve by observing some of the points on it. The practical man who looked only to immediate utility would merely observe the points he required for some special object; these points would be badly distributed on the curve, they would be crowded together in certain parts and scarce in others, so that it would be impossible to connect them by a continuous line, and they would be useless for any other application. The scientist would proceed in a different manner. Since he wishes to study the curve for itself, he will distribute the points to be observed regularly, and as soon as he knows some of them, he will join them by a regular line, and he will then have the complete curve. But how is he to accomplish this? If he has determined one extreme point on the curve, he will not remain close to this extremity, but will move to the other end. After the two extremities, the central point is the most instructive, and so on.\\
因此,我们应该从规律性的事实入手;但一旦规律得以确立,一旦它不再受到质疑,那些完全符合该规律的事实就失去了吸引力,因为它们无法再教给我们新的东西。这时,例外情况就变得重要起来。我们不再寻找相似之处,而是首先关注差异,并且在这些差异中,我们会优先选择那些最为突出的差异,这不仅是因为它们最为引人注目,还因为它们最具启发性。一个简单的例子就能很好地说明这一点。假设我们试图通过观察曲线上的一些点来确定这条曲线。只看重即时效用的实用主义者只会观察那些为了某个特定目的所需的点;这些点在曲线上分布得很不均匀,在某些部分过于密集,而在其他部分则很稀少,以至于无法用一条连续的线将它们连接起来,而且对于其他任何应用来说,这些点也毫无用处。科学家则会采用不同的方法。由于他希望研究曲线本身,他会有规律地分布要观察的点,一旦知道了其中一些点,他就会用一条规则的线将它们连接起来,从而得到完整的曲线。但他是如何做到这一点的呢?如果他确定了曲线上的一个端点,他不会一直停留在这个端点附近,而是会移到另一端。确定了两个端点之后,中心点就是最具启发性的,以此类推。 \\

\noindent 12\\
Thus when a rule has been established, we have first to look for the cases in which the rule stands the best chance of being found in fault. This is one of many reasons for the interest of astronomical facts and of geological ages. By making long excursions in space or in time, we may find our ordinary rules completely upset, and these great upsettings will give us a clearer view and better comprehension of such small changes as may occur nearer us, in the small corner of the world in which we are called to live and move. We shall know this corner better for the journey we have taken into distant lands where we had no concern.\\
因此,当一条规律确立之后,我们首先要寻找那些最有可能发现该规律存在问题的情况。这就是天文学事实和地质年代引人关注的众多原因之一。通过在空间或时间上进行 “长途探索”,我们可能会发现我们日常的规律被彻底颠覆,而这些重大的颠覆会让我们更清晰地认识并更好地理解那些发生在我们身边的细微变化,也就是在我们生活和活动的这个小世界里的变化。通过 “前往” 那些原本与我们无关的遥远 “领域” 探索,我们会对自己所处的这个小世界有更深入的了解。 \\

\noindent 13\\
But what we must aim at is not so much to ascertain resemblances and differences, as to discover similarities hidden under apparent discrepancies. The individual rules appear at first discordant, but on looking closer we can generally detect a resemblance; though differing in matter, they approximate in form and in the order of their parts. When we examine them from this point of view, we shall see them widen and tend to embrace everything. This is what gives a value to certain facts that come to complete a whole, and show that it is the faithful image of other known wholes.\\
然而,我们的主要目标与其说是确定相似点和不同点,不如说是发现隐藏在明显差异之下的相似性。各个单独的规律乍看之下并不协调,但仔细观察,我们通常能发现它们之间的相似之处;尽管在内容上有所不同,但它们在形式和组成部分的排列顺序上相近。当我们从这个角度去审视它们时,就会发现这些规律的涵盖范围会扩大,并趋向于包罗万象。这就是某些能够构成一个整体,并表明该整体是其他已知整体的真实写照的事实所具有的价值所在。 v

\noindent 14\\
I cannot dwell further on this point, but these few words will suffice to show that the scientist does not make a random selection of the facts to be observed. He does not count lady - birds, as Tolstoi says, because the number of these insects, interesting as they are, is subject to capricious variations. He tries to condense a great deal of experience and a great deal of thought into a small volume, and that is why a little book on physics contains so many past experiments, and a thousand times as many possible ones, whose results are known in advance.\\
关于这一点我不再赘述,但仅这几句话就足以表明,科学家并非随意选择要观察的事实。正如托尔斯泰所说,科学家不会去数瓢虫的数量,因为这些昆虫的数量尽管有趣,却会发生无规律的变化。科学家试图将大量的经验和思考浓缩在一个小的 “体量” 中,这就是为什么一本薄薄的物理学书籍中包含了如此多以往的实验,以及数量是其千倍之多的、结果已预先知晓的潜在实验。 \\

\noindent 15\\
But so far we have only considered one side of the question. The scientist does not study nature because it is useful to do so. He studies it because he takes pleasure in it, and he takes pleasure in it because it is beautiful. If nature were not beautiful it would not be worth knowing, and life would not be worth living. I am not speaking, of course, of that beauty which strikes the senses, of the beauty of qualities and appearances. I am far from despising this, but it has nothing to do with science. What I mean is that more intimate beauty which comes from the harmonious order of its parts, and which a pure intelligence can grasp. It is this that gives a body a skeleton, so to speak, to the shimmering visions that flatter our senses, and without this support the beauty of these fleeting dreams would be imperfect, because it would be indefinite and ever elusive. Intellectual beauty, on the contrary, is self-sufficing, and it is for it, more perhaps than for the future good of humanity, that the scientist condemns himself to long and painful labours.\\
但到目前为止,我们只考虑了问题的一个方面。科学家研究自然,并非因为这样做有用。他们研究自然是因为从中获得乐趣,而他们能获得乐趣是因为自然是美的。如果自然不美丽,那就不值得去了解,生活也不值得去过。当然,我所说的美,不是那种诉诸感官的美,不是表象和特质之美。我绝不是轻视这种美,只是它与科学无关。我指的是一种更为内在的美,它源自自然各部分之间和谐的秩序,是纯粹的理性能够理解的美。可以说,正是这种美为那些取悦我们感官的虚幻景象赋予了 “骨架”,如果没有这种支撑,这些转瞬即逝的幻景之美就是不完美的,因为它会变得模糊且难以捉摸。相反,理性之美是自足的,或许正是为了这种美,而不仅仅是为了人类未来的福祉,科学家们才甘愿投身于漫长而艰苦的研究工作中。 \\

\noindent 16\\
It is, then, the search for this special beauty, the sense of the harmony of the world, that makes us select the facts best suited to contribute to this harmony; just as the artist selects those features of his sitter which complete the portrait and give it character and life. And there is no fear that this instinctive and unacknowledged preoccupation will divert the scientist from the search for truth. We may dream of a harmonious world, but how far it will fall short of the real world! The Greeks, the greatest artists that ever were, constructed a heaven for themselves; how poor a thing it is beside the heaven as we know it!\\
因此,正是对这种独特的美,即对世界和谐的感知的追求,促使我们选择最能促成这种和谐的事实;就如同艺术家挑选模特身上那些能使肖像画更加完整,并赋予其个性与生命力的特征一样。而且不必担心这种本能的、潜意识里的关注会让科学家偏离对真理的追求。我们或许会憧憬一个和谐的世界,但它与现实世界的差距何其大!古希腊人,有史以来最伟大的艺术家,为自己构建了一个天堂;但与我们所知的现实的 “天堂” 相比,那是多么微不足道啊! \\

\noindent 17\\
It is because simplicity and vastness are both beautiful that we seek by preference simple facts and vast facts; that we take delight, now in following the giant courses of the stars, now in scrutinizing with a microscope that prodigious smallness which is also a vastness, and now in seeking in geological ages the traces of a past that attracts us because of its remoteness.\\
正是因为简洁与宏大皆具美感,我们才会优先探寻简单的事实和宏大的事实;我们时而沉醉于追寻恒星的浩瀚轨迹,时而着迷于用显微镜仔细观察那同样广袤无垠的微观世界,时而热衷于在地质年代中寻觅因遥远而吸引我们的过往痕迹。 \\

\noindent 18\\
Thus we see that care for the beautiful leads us to the same selection as care for the useful. Similarly economy of thought, that economy of effort which, according to Mach, is the constant tendency of science, is a source of beauty as well as a practical advantage. The buildings we admire are those in which the architect has succeeded in proportioning the means to the end, in which the columns seem to carry the burdens imposed on them lightly and without effort, like the graceful caryatids of the Erechtheum.\\
因此我们看到,对美的追求与对实用性的追求,会引导我们做出相同的选择。同样,思维的经济性,即马赫所说的科学一直以来的发展趋势——精力的节省,它既是美感的来源,也具有实际的益处。我们所欣赏的建筑,是那些建筑师成功地使手段与目的相匹配的建筑,在这样的建筑中,柱子似乎毫不费力地承担着施加在它们身上的重量,就像厄瑞克忒翁神庙中优雅的女像柱一样。 \\ 

\noindent 19\\
Whence comes this concordance? Is it merely that things which seem to us beautiful are those which are best adapted to our intelligence, and that consequently they are at the same time the tools that intelligence knows best how to handle? Or is it due rather to evolution and natural selection? Have the peoples whose ideal conformed best to their own interests, properly understood, exterminated the others and taken their place? One and all pursued their ideal without considering the consequences, but while this pursuit led some to their destruction, it gave empire to others. We are tempted to believe this, for if the Greeks triumphed over the barbarians, and if Europe, heir of the thought of the Greeks, dominates the world, it is due to the fact that the savages loved garish colours and the blatant noise of the drum, which appealed to their senses, while the Greeks loved the intellectual beauty hidden behind sensible beauty, and that it is this beauty which gives certainty and strength to the intelligence.\\
这种一致性从何而来?仅仅是因为在我们看来美的事物,是那些最适合我们智力理解的事物,因此它们同时也是我们的智力最擅长运用的 “工具” 吗?还是说这更多地归因于进化和自然选择呢?那些理想最契合自身真正利益的民族,是否消灭了其他民族并取而代之?所有人都在追求自己的理想,而不顾及后果,但这种追求让一些民族走向毁灭,却让另一些民族得以主宰。我们不禁会这么认为,因为倘若希腊人战胜了野蛮人,而继承了希腊思想的欧洲主宰了世界,那是由于野蛮人喜爱鲜艳刺眼的色彩和震耳的鼓声,这些诉诸他们的感官;而希腊人喜爱隐藏在感官之美背后的理性之美,正是这种美赋予了智力确定性与力量。\\ 

\noindent 20\\
No doubt Tolstoi would be horrified at such a triumph, and he would refuse to admit that it could be truly useful. But this disinterested pursuit of truth for its own beauty is also wholesome, and can make men better. I know very well there are disappointments, that the thinker does not always find the serenity he should, and even that some scientists have thoroughly bad tempers. \\\\
毫无疑问,托尔斯泰会对这样的 “胜利” 感到震惊,而且他会拒绝承认这真的有益。但是,这种为了真理本身的美而进行的无私追求也是有益健康的,并且能够使人变得更好。我深知其中会有失望,思想家并不总是能获得应有的宁静,甚至有些科学家脾气还极其糟糕。 \\

\noindent 21\\
Must we therefore say that science should be abandoned, and morality alone be studied? Does anyone suppose that moralists themselves are entirely above reproach when they have come down from the pulpit?\\
那么,我们是否就该说应当放弃科学,只研究道德呢?又有谁会认为,那些道德学家走下讲台之后,自身就完全无可指责了呢? \\

\begin{center}
III. MATHEMATICAL DISCOVERY 数学发现 
\end{center}

\noindent 1\\
The genesis of mathematical discovery is a problem which must inspire the psychologist with the keenest interest. For this is the process in which the human mind seems to borrow least from the exterior world, in which it acts, or appears to act, only by itself and on itself, so that by studying the process of geometric thought we may hope to arrive at what is most essential in the human mind.\\
数学发现的起源是一个必定会引起心理学家浓厚兴趣的问题。因为在这个过程中,人类的思维似乎从外部世界借鉴得最少,在其中,思维仅仅依靠自身并作用于自身来运作,或者看似如此。因此,通过研究几何思维的过程,我们或许有望洞悉人类思维中最为本质的东西。 \\

\noindent 2\\
This has long been understood, and a few months ago a review called *L'Enseignement mathématique*, edited by MM. Laisant and Fehr, instituted an enquiry into the habits of mind and methods of work of different mathematicians. I had outlined the principal features of this article when the results of the enquiry were published, so that I have hardly been able to make any use of them, and I will content myself with saying that the majority of the evidence confirms my conclusions. I do not say there is unanimity, for on an appeal to universal suffrage we cannot hope to obtain unanimity.\\
这一点早已为人所理解。几个月前,由莱桑(Laisant)先生和费尔(Fehr)先生编辑的名为《数学教育》(*L'Enseignement mathématique*)的评论刊物,对不同数学家的思维习惯和工作方法展开了一项调查。在该调查结果公布时,我已经勾勒出了这篇文章的主要要点,所以几乎没能用上这些调查结果。我只想说,大多数的调查证据证实了我的结论。我并不是说大家意见完全一致,因为在进行全体投票表决时,我们不能指望达成完全的共识。\\  

\noindent 3\\
One first fact must astonish us, or rather would astonish us if we were not too much accustomed to it. How does it happen that there are people who do not understand mathematics? If the science invokes only the rules of logic, those accepted by all well - formed minds, if its evidence is founded on principles that are common to all men, and that none but a madman would attempt to deny, how does it happen that there are so many people who are entirely impervious to it?\\
首先有一个事实一定会让我们感到惊讶,或者说,如果我们不是对此太过习以为常的话,本应会感到惊讶。怎么会有人不懂数学呢?如果这门学科只运用所有思维正常的人都认可的逻辑规则,如果其论据建立在所有人都认同、只有疯子才会试图否认的原则之上,那为什么会有这么多人对数学完全一窍不通呢? \\

\noindent 4\\
There is nothing mysterious in the fact that everyone is not capable of discovery. That everyone should not be able to retain a demonstration he has once learnt is still comprehensible. But what does seem most surprising, when we consider it, is that anyone should be unable to understand a mathematical argument at the very moment it is stated to him. And yet those who can only follow the argument with difficulty are in a majority; this is incontestable, and the experience of teachers of secondary education will certainly not contradict me.\\
并非每个人都具备发现能力,这一事实并不神秘。每个人不能记住曾经学过的数学证明,这也仍然可以理解。但当我们细想时,最令人惊讶的是,竟然有人在别人向他阐述数学论证的当下,却无法理解。然而,大多数人理解数学论证都有困难,这是无可争辩的,而且中学教师的教学经验肯定也能证实这一点。 \\

\noindent 5\\
And still further, how is error possible in mathematics? A healthy intellect should not be guilty of any error in logic, and yet there are very keen minds which will not make a false step in a short argument such as those we have to make in the ordinary actions of life, which yet are incapable of following or repeating without error the demonstrations of mathematics which are longer, but which are, after all, only accumulations of short arguments exactly analogous to those they make so easily. Is it necessary to add that mathematicians themselves are not infallible?\\
更进一步说,数学中怎么会出现错误呢?一个健全的心智在逻辑上本不应犯错,然而,有些极为敏锐的人,在日常生活中的简短推理中不会出错,却无法不出错地理解或复述较长的数学证明,可这些较长的数学证明归根结底也只是由一些与他们轻易就能完成的简短推理完全类似的内容累积而成的。还需要补充说明的是,数学家本身也并非绝对正确吧?\\ 

\noindent 6\\
The answer appears to me obvious. Imagine a long series of syllogisms in which the conclusions of those that precede form the premises of those that follow. We shall be capable of grasping each of the syllogisms, and it is not in the passage from premises to conclusion that we are in danger of going astray. But between the moment when we meet a proposition for the first time as the conclusion of one syllogism, and the moment when we find it once more as the premise of another syllogism, much time will sometimes have elapsed, and we shall have unfolded many links of the chain; accordingly it may well happen that we shall have forgotten it, or, what is more serious, forgotten its meaning. So we may chance to replace it by a somewhat different proposition, or to preserve the same statement but give it a slightly different meaning, and thus we are in danger of falling into error.\\\\
在我看来,答案显而易见。设想有一长串的三段论,前面三段论的结论构成后面三段论的前提。我们能够理解每一个三段论,而且在从前提到结论的推导过程中,我们不太容易出错。但是,从我们第一次遇到一个命题作为某一个三段论的结论,到我们再次发现它作为另一个三段论的前提,这中间有时会间隔很长时间,而且我们可能已经推导了这个逻辑链条上的很多环节;因此很可能出现的情况是,我们已经忘记了这个命题,或者更严重的是,忘记了它的含义。所以我们可能会无意中用一个略有不同的命题来取代它,或者使用同样的表述但赋予其稍有不同的含义,这样一来,我们就有陷入错误的风险。\\ 

\noindent 7\\
A mathematician must often use a rule, and, naturally, he begins by demonstrating the rule. At the moment the demonstration is quite fresh in his memory he understands perfectly its meaning and significance, and he is in no danger of changing it. But later on he commits it to memory, and only applies it in a mechanical way, and then, if his memory fails him, he may apply it wrongly. It is thus, to take a simple and almost vulgar example, that we sometimes make mistakes in calculation, because we have forgotten our multiplication table.\\
数学家常常需要运用某种规则,自然地,他一开始会先证明这个规则。当证明过程还在他的记忆中十分清晰的时候,他完全理解其含义和重要性,也不会弄错。但后来他记住了这个规则,只是机械地运用它,这时,如果他记错了,就可能会用错。举一个简单甚至有些平常的例子,我们有时会在计算中出错,就是因为忘了乘法口诀表。 \\

\noindent 8\\
On this view special aptitude for mathematics would be due to nothing but a very certain memory or a tremendous power of attention. It would be a quality analogous to that of the whist player who can remember the cards played, or, to rise a step higher, to that of the chess player who can picture a very great number of combinations and retain them in his memory. Every good mathematician should also be a good chess player and vice versa, and similarly he should be a good numerical calculator. Certainly this sometimes happens, and thus Gauss was at once a geometrician of genius and a very precocious and very certain calculator.\\
按照这种观点,对数学的特殊天赋仅仅源于超强的记忆力或高度集中的注意力。这就如同玩惠斯特牌的人能记住出过的牌,或者更高层次一点,像棋手能在脑海中想象出大量的棋路组合并记住它们一样。每一位优秀的数学家也应该是优秀的棋手,反之亦然,同样,他也应该是出色的计算能手。当然,这种情况有时确实存在,比如高斯既是一位天才的几何学家,也是一位早熟且计算能力极强的人。 \\

\noindent 9\\
But there are exceptions, or rather I am wrong, for I cannot call them exceptions, otherwise the exceptions would be more numerous than the cases of conformity with the rule. On the contrary, it was Gauss who was an exception. As for myself, I must confess I am absolutely incapable of doing an addition sum without a mistake. Similarly I should be a very bad chess player. I could easily calculate that by playing in a certain way I should be exposed to such and such a danger; I should then review many other moves, which I should reject for other reasons, and I should end by making the move I first examined, having forgotten in the interval the danger I had foreseen.\\
但也有例外情况,或者更确切地说,是我错了,因为我不能称它们为例外,否则例外的情况会比符合规律的情况还要多。相反,高斯才是个例外。至于我自己,必须承认我做加法运算时绝对会出错。同样,我也会是个很差劲的棋手。我能轻易算出,如果以某种方式落子,就会面临这样或那样的危险;然后我会考虑许多其他走法,又因为其他原因否定它们,最后我还是会走出最初考虑的那一步,却在这个过程中忘记了之前预见到的危险。 \\

\noindent 10\\
In a word, my memory is not bad, but it would be insufficient to make me a good chess player. Why, then, does it not fail me in a difficult mathematical argument in which the majority of chess players would be lost? Clearly because it is guided by the general trend of the argument. A mathematical demonstration is not a simple juxtaposition of syllogisms; it consists of syllogisms placed in a certain order, and the order in which these elements are placed is much more important than the elements themselves. If I have the feeling, so to speak the intuition, of this order, so that I can perceive the whole of the argument at a glance, I need no longer be afraid of forgetting one of the elements; each of them will place itself naturally in the position prepared for it, without my having to make any effort of memory.\\
总之,我的记忆力不算差,但还不足以让我成为一名优秀的棋手。那么,为什么在大多数棋手都会感到困惑的复杂数学论证中,我的记忆力却不会出问题呢?显然是因为它受到论证总体思路的引导。数学证明不是三段论的简单罗列,它是由按照特定顺序排列的三段论组成的,而且这些组成部分的排列顺序远比其本身重要得多。可以说,如果我对这种顺序有感觉,即具备直觉,从而能够一眼看清整个论证的全貌,那我就不用担心会忘记其中的某个部分;每一部分都会自然而然地出现在为它准备好的位置上,而无需我刻意去记忆。 \\

\noindent 11\\
It seems to me, then, as I repeat an argument I have learnt, that I could have discovered it. This is often only an illusion; but even then, even if I am not clever enough to create for myself, I rediscover it myself as I repeat it.
因此,当我复述一个学过的论证时,在我看来,我本可以自己发现它。这往往只是一种错觉;但即便如此,即使我不够聪明,无法独立创造出这个论证,在复述的过程中,我也像是重新发现了它。 \\ 

\noindent 12\\
We can understand that this feeling, this intuition of mathematical order, which enables us to guess hidden harmonies and relations, cannot belong to everyone. Some have neither this delicate feeling that is difficult to define, nor a power of memory and attention above the common, and so they are absolutely incapable of understanding even the first steps of higher mathematics. This applies to the majority of people. Others have the feeling only in a slight degree, but they are gifted with an uncommon memory and a great capacity for attention. They learn the details one after the other by heart, they can understand mathematics and sometimes apply them, but they are not in a condition to create. Lastly, others possess the special intuition I have spoken of more or less highly developed, and they cannot only understand mathematics, even though their memory is in no way extraordinary, but they can become creators, and seek to make discovery with more or less chance of success, according as their intuition is more or less developed.\\
我们能够理解,这种对数学秩序的感觉和直觉,能够让我们猜测到隐藏的和谐与关系,但并非每个人都具备。有些人既没有这种难以言喻的微妙感觉,也没有超出常人的记忆力和注意力,因此他们甚至完全无法理解高等数学的入门知识,这适用于大多数人。另一些人仅有微弱的这种感觉,但他们拥有非凡的记忆力和很强的注意力。他们能逐个记住细节,能够理解数学知识,有时还能应用这些知识,但他们不具备创造性。最后,还有一些人或多或少高度具备我所说的这种特殊直觉,即便他们的记忆力并不出众,也不仅能够理解数学,还能成为创造者,并根据自身直觉的发展程度,在不同程度上有机会取得数学发现。 \\

\noindent 13\\
What, in fact, is mathematical discovery? It does not consist in making new combinations with mathematical entities that are already known. That can be done by anyone, and the combinations that could be so formed would be infinite in number, and the greater part of them would be absolutely devoid of interest. Discovery consists precisely in not constructing useless combinations, but in constructing those that are useful, which are an infinitely small minority. Discovery is discernment, selection.\\
事实上,什么是数学发现呢?它并不在于用已知的数学元素进行新的组合。任何人都能做到这一点,而且这样形成的组合数量是无穷无尽的,其中大部分组合毫无价值。数学发现恰恰在于不构建无用的组合,而是构建那些有用的组合,而有用的组合只占极少数。发现即辨别和选择。 \\

\noindent 14\\
How this selection is to be made I have explained above. Mathematical facts worthy of being studied are those which, by their analogy with other facts, are capable of conducting us to the knowledge of a mathematical law, in the same way that experimental facts conduct us to the knowledge of a physical law. They are those which reveal unsuspected relations between other facts, long since known, but wrongly believed to be unrelated to each other.\\
关于如何进行这种选择,我在上面已经解释过了。值得研究的数学事实,是那些通过与其他事实的相似性,能够引导我们认识数学规律的事实,就如同实验事实引导我们认识物理规律一样。这些数学事实揭示了那些早已为人所知,但一直被错误认为彼此毫无关联的事实之间,意想不到的联系。 \\ 

\noindent 15\\
Among the combinations we choose, the most fruitful are often those which are formed of elements borrowed from widely separated domains. I do not mean to say that for discovery it is sufficient to bring together objects that are as incongruous as possible. The greater part of the combinations so formed would be entirely fruitless, but some among them, though very rare, are the most fruitful of all.\\
在我们所选择的组合中,最有成效的往往是那些由来自截然不同领域的元素构成的组合。我并不是说,只要把尽可能不相关的对象组合在一起就足以实现数学发现。这样形成的大多数组合完全没有成果,但其中有一些(尽管极为罕见)却是最富有成效的。 \\ 

\noindent 16\\
Discovery, as I have said, is selection. But this is perhaps not quite the right word. It suggests a purchaser who has been shown a large number of samples, and examines them one after the other in order to make his selection. In our case the samples would be so numerous that a whole life would not give sufficient time to examine them. Things do not happen in this way. Unfruitful combinations do not so much as present themselves to the mind of the discoverer. In the field of his consciousness there never appear any but really useful combinations, and some that he rejects, which, however, partake to some extent of the character of useful combinations. Everything happens as if the discoverer were a secondary examiner who had only to interrogate candidates declared eligible after passing a preliminary test.\\
正如我所说,发现即选择。但这个词或许不太准确。它让人联想到一个买家,面前有大量的样品展示,然后他逐一查看以便做出选择。但在我们所说的数学发现中,样品数量如此之多,以至于用尽一生的时间也不足以全部检验。事情并非如此发展。无成效的组合甚至都不会出现在发现者的脑海中。在他的意识领域里,出现的只有真正有用的组合,以及一些他摒弃的组合,不过这些被摒弃的组合在某种程度上也具备有用组合的特征。这一切就好像发现者是一个复试考官,只需要对通过初试的合格候选人进行考核一样 。 \\

\noindent 17\\
But what I have said up to now is only what can be observed or inferred by reading the works of geometricians, provided they are read with some reflection.\\
但到目前为止我所说的,仅仅是通过阅读几何学家的著作,并加以思考后,所能观察到或推断出的内容。\\ 

\noindent 18\\
It is time to penetrate further, and to see what happens in the very soul of the mathematician. For this purpose I think I cannot do better than recount my personal recollections. Only I am going to confine myself to relating how I wrote my first treatise on Fuchsian functions. I must apologize, for I am going to introduce some technical expressions, but they need not alarm the reader, for he has no need to understand them. I shall say, for instance, that I found the demonstration of such and such a theorem under such and such circumstances; the theorem will have a barbarous name that many will not know, but that is of no importance. What is interesting for the psychologist is not the theorem but the circumstances.\\
是时候进一步深入探究,看看数学家内心究竟发生了什么。为此,我认为最好的办法莫过于讲述我的个人回忆。我只打算讲讲我是如何撰写关于富克斯函数的第一篇论文的。我得先道个歉,因为我会用到一些专业术语,但读者不必担心,因为他们无需理解这些术语。例如,我会说在这样或那样的情况下,我得出了某个定理的证明;这个定理可能有个很生僻的名字,很多人都没听说过,但这并不重要。对心理学家来说,有趣的不是定理本身,而是得出定理的相关情境。\\ 

\noindent 19\\
For a fortnight I had been attempting to prove that there could not be any function analogous to what I have since called Fuchsian functions. I was at that time very ignorant. Every day I sat down at my table and spent an hour or two trying a great number of combinations, and I arrived at no result. One night I took some black coffee, contrary to my custom, and was unable to sleep. A host of ideas kept surging in my head; I could almost feel them jostling one another, until two of them coalesced, so to speak, to form a stable combination. When morning came, I had established the existence of one class of Fuchsian functions, those that are derived from the hyper - geometric series. I had only to verify the results, which only took a few hours.\\
有两周时间,我一直试图证明不存在类似于我后来称之为富克斯函数的函数。那时我所知甚少。每天我都坐在桌前,花一两个小时尝试大量的组合,但都毫无结果。一天晚上,我一反往常喝了些黑咖啡,结果无法入睡。无数的想法在我脑海中不断涌现,我几乎能感觉到它们在相互碰撞,直到其中两个想法可以说融合在一起,形成了一个稳定的组合。到了早上,我证明了一类富克斯函数的存在,即那些从超几何级数推导出来的函数。我只需要花几个小时来验证结果。 \\

\noindent 20\\
Then I wished to represent these functions by the quotient of two series. This idea was perfectly conscious and deliberate; I was guided by the analogy with elliptical functions. I asked myself what must be the properties of these series, if they existed, and I succeeded without difficulty in forming the series that I have called Theta - Fuchsian.\\
随后,我希望用两个级数的商来表示这些函数。这个想法是完全有意识且经过深思熟虑的,我是受到了椭圆函数的启发。我问自己,如果这些级数存在,那它们应该具备哪些性质,然后我顺利地构造出了我称之为“θ - 富克斯”的级数。 \\

\noindent 21\\
At this moment I left Caen, where I was then living, to take part in a geological conference arranged by the School of Mines. The incidents of the journey made me forget my mathematical work. When we arrived at Coutances, we got into a break to go for a drive, and, just as I put my foot on the step, the idea came to me, though nothing in my former thoughts seemed to have prepared me for it, that the transformations I had used to define Fuchsian functions were identical with those of non - Euclidian geometry. I made no verification, and had no time to do so, since I took up the conversation again as soon as I had sat down in the break, but I felt absolute certainty at once. When I got back to Caen I verified the result at my leisure to satisfy my conscience.\\
这时,我离开当时居住的卡昂,去参加矿业学院组织的一次地质会议。旅途中的种种事情让我暂时忘记了数学研究工作。抵达库唐斯后,我们中途休息去兜风,就在我踏上踏板的那一刻,我突然想到,我用来定义富克斯函数的变换与非欧几里得几何的变换是相同的,尽管我之前的思考中似乎没有任何内容为此作铺垫。我没有立即验证,也没有时间验证,因为一坐下休息我就又聊起天来,但我立刻就对此深信不疑。回到卡昂后,我抽空验证了这个结果,以求心安。\\ 

\noindent 22\\
I then began to study arithmetical questions without any great apparent result, and without suspecting that they could have the least connexion with my previous researches. Disgusted at my want of success, I went away to spend a few days at the seaside, and thought of entirely different things. One day, as I was walking on the cliff, the idea came to me, again with the same characteristics of conciseness, suddenness, and immediate certainty, that arithmetical transformations of indefinite ternary quadratic forms are identical with those of non - Euclidian geometry.\\
随后,我开始研究算术问题,但显然没有取得什么重大成果,而且我也没想到这些问题与我之前的研究有丝毫关联。因未能成功而感到沮丧,我去海边待了几天,想些完全不同的事情。有一天,我在悬崖上散步时,突然又有了一个想法,这个想法同样简洁、突然且让我立刻确信:不定三元二次型的算术变换与非欧几里得几何的变换是相同的。 \\

\noindent 23\\
Returning to Caen, I reflected on this result and deduced its consequences. The example of quadratic forms showed me that there are Fuchsian groups other than those which correspond with the hyper - geometric series; I saw that I could apply to them the theory of the Theta - Fuchsian series, and that, consequently, there are Fuchsian functions other than those which are derived from the hyper - geometric series, the only ones I knew up to that time. Naturally, I proposed to form all these functions. I laid siege to them systematically and captured all the outworks one after the other. There was one, however, which still held out, whose fall would carry with it that of the central fortress. But all my efforts were of no avail at first, except to make me better understand the difficulty, which was already something. All this work was perfectly conscious.\\
回到卡昂后,我思考了这个结果,并推导出其影响。二次型的例子让我明白,除了与超几何级数相对应的富克斯群之外,还有其他的富克斯群;我意识到可以将“θ - 富克斯”级数理论应用于它们,因此,除了我当时已知的由超几何级数推导出来的富克斯函数之外,还存在其他的富克斯函数。自然而然地,我打算构造出所有这些函数。我系统地研究它们,逐个攻克了所有次要的问题。然而,有一个问题仍然难以解决,一旦攻克它,核心问题也将迎刃而解。但起初我所有的努力都徒劳无功,不过这让我更好地理解了其中的难点,这也算有所收获。所有这些工作都是在完全有意识的状态下进行的。 \\

\noindent 24\\
Thereupon I left for Mont - Valérien, where I had to serve my time in the army, and so my mind was preoccupied with very different matters. One day, as I was crossing the street, the solution of the difficulty which had brought me to a standstill came to me all at once. I did not try to fathom it immediately, and it was only after my service was finished that I returned to the question. I had all the elements, and had only to assemble and arrange them. Accordingly I composed my definitive treatise at a sitting and without any difficulty.\\
于是我前往瓦莱里昂山服兵役,那时我的心思都放在了截然不同的事情上。有一天,我过马路的时候,那个曾让我陷入僵局的难题的解决方案突然出现在我的脑海中。我没有立刻深入探究,直到兵役结束后,我才重新思考这个问题。我已经掌握了所有的要素,只需要将它们整合和梳理。于是,我一气呵成,毫无困难地完成了最终的论文。 \\

\noindent 25\\
It is useless to multiply examples, and I will content myself with this one alone. As regards my other researches, the accounts I should give would be exactly similar, and the observations related by other mathematicians in the enquiry of *L'Enseignement mathématique* would only confirm them.\\
再多举例子也无必要,我仅以这一个为例就足够了。至于我的其他研究,讲述起来情况也会大致相同,而且其他数学家在《数学教育》杂志的调查中所分享的见解,也只会进一步证实这些观点。 \\ 

\noindent 26\\
One is at once struck by these appearances of sudden illumination, obvious indications of a long course of previous unconscious work. The part played by this unconscious work in mathematical discovery seems to me indisputable, and we shall find traces of it in other cases where it is less evident. Often when a man is working at a difficult question, he accomplishes nothing the first time he sets to work. Then he takes more or less of a rest, and sits down again at his table. During the first half - hour he still finds nothing, and then all at once the decisive idea presents itself to his mind. We might say that the conscious work proved more fruitful because it was interrupted and the rest restored force and freshness to the mind. But it is more probable that the rest was occupied with unconscious work, and that the result of this work was afterwards revealed to the geometrician exactly as in the cases I have quoted, except that the revelation, instead of coming to light during a walk or a journey, came during a period of conscious work, but independently of that work, which at most only performs the unlocking process, as if it were the spur that excited into conscious form the results already acquired during the rest, which till then remained unconscious.\\
人们会立刻被这些突然顿悟的现象所吸引,这显然表明在此之前有过长期的无意识思考过程。在我看来,无意识思考在数学发现中所起的作用是毋庸置疑的,而且我们会在其他一些不太明显的例子中找到它的踪迹。通常,当一个人钻研一个难题时,他一开始往往毫无进展。然后他或多或少休息一下,再回到桌前。在开始的半个小时里,他仍然一无所获,但随后,决定性的想法会突然出现在他的脑海中。我们或许可以说,有意识的思考之所以更有成效,是因为它被打断了,而休息让大脑恢复了精力和活力。但更有可能的是,休息期间大脑在进行无意识的思考,而且这种思考的结果会像我所举的例子那样,随后呈现在几何学家的脑海中,只不过这种顿悟并非发生在散步或旅行时,而是出现在有意识的思考期间,不过与有意识的思考本身并无关联,有意识的思考至多只是起到了启发的作用,就好像是它促使在休息时获得但一直处于无意识状态的结果,以有意识的形式呈现出来。\\ 

\noindent 27\\
There is another remark to be made regarding the conditions of this unconscious work, which is, that it is not possible, or in any case not fruitful, unless it is first preceded and then followed by a period of conscious work. These sudden inspirations are never produced (and this is sufficiently proved already by the examples I have quoted) except after some days of voluntary efforts which appeared absolutely fruitless, in which one thought one had accomplished nothing, and seemed to be on a totally wrong track. These efforts, however, were not as barren as one thought; they set the unconscious machine in motion, and without them it would not have worked at all, and would not have produced anything.\\
关于这种无意识思考的条件,还有一点需要说明,那就是如果没有在其之前和之后进行有意识的思考,无意识思考要么不可能发生,要么无论如何都不会有成果。这些突然的灵感从来不会凭空出现(我所举的例子已经充分证明了这一点), 而是要经过数天看似完全徒劳的主动思考,在这期间,人们以为自己一无所获,仿佛完全走错了方向。然而,这些努力并不像人们以为的那样毫无意义,它们启动了无意识思考的 “机器”,没有这些努力,这台 “机器” 根本不会运转,也不会产生任何成果。 \\

\noindent 28\\
The necessity for the second period of conscious work can be even more readily understood. It is necessary to work out the results of the inspiration, to deduce the immediate consequences and put them in order and to set out the demonstrations; but, above all, it is necessary to verify them. I have spoken of the feeling of absolute certainty which accompanies the inspiration; in the cases quoted this feeling was not deceptive, and more often than not this will be the case. But we must beware of thinking that this is a rule without exceptions. Often the feeling deceives us without being any less distinct on that account, and we only detect it when we attempt to establish the demonstration. I have observed this fact most notably with regard to ideas that have come to me in the morning or at night when I have been in bed in a semi-somnolent condition.\\
第二阶段有意识思考的必要性则更易于理解。我们需要深入研究灵感带来的成果,推导出直接的结论,将它们梳理清楚,并给出证明;但最重要的是,必须对其进行验证。我曾提到过,灵感乍现时会伴随着一种绝对确信的感觉;在我所举的例子中,这种感觉并未误导我,而且在多数情况下也是如此。但我们不能就此认为这是一条毫无例外的规律。这种感觉常常会误导我们,却又同样强烈,只有在我们试图进行论证时,才会察觉到被误导。我尤其注意到,清晨或夜晚,当我处于半梦半醒的状态时产生的那些想法,就会出现这种情况。 \\ 

\noindent 29\\
Such are the facts of the case, and they suggest the following reflections. The result of all that precedes is to show that the unconscious ego, or, as it is called, the subliminal ego, plays a most important part in mathematical discovery. But the subliminal ego is generally thought of as purely automatic. Now we have seen that mathematical work is not a simple mechanical work, and that it could not be entrusted to any machine, whatever the degree of perfection we suppose it to have been brought to. It is not merely a question of applying certain rules, of manufacturing as many combinations as possible according to certain fixed laws. The combinations so obtained would be extremely numerous, useless, and encumbering. The real work of the discoverer consists in choosing between these combinations with a view to eliminating those that are useless, or rather not giving himself the trouble of making them at all. The rules which must guide this choice are extremely subtle and delicate, and it is practically impossible to state them in precise language; they must be felt rather than formulated. Under these conditions, how can we imagine a sieve capable of applying them mechanically?\\
情况就是如此,由此可以引发以下思考。上述所有内容都表明,无意识的自我,或者说潜意识自我,在数学发现中起着至关重要的作用。但潜意识自我通常被认为是完全自动运行的。而我们已经看到,数学研究并非简单的机械性工作,不能将其交给任何机器去完成,无论我们认为这台机器有多么完善。这不仅仅是运用某些规则,按照特定的固定法则尽可能多地构造组合的问题。这样得到的组合数量极多,却毫无用处,还会带来困扰。发现者的真正工作在于在这些组合中做出选择,以剔除那些无用的组合,或者说根本就不去费心构造它们。指导这种选择的规则极其微妙,实际上几乎不可能用精确的语言表述出来,人们必须去感受这些规则,而不是将它们公式化。在这种情况下,我们怎么能想象出一台能够机械地运用这些规则的 “筛选器” 呢? \\

\noindent 30\\
The following, then, presents itself as a first hypothesis. The subliminal ego is in no way inferior to the conscious ego; it is not purely automatic; it is capable of discernment; it has tact and lightness of touch; it can select, and it can divine. More than that, it can divine better than the conscious ego, since it succeeds where the latter fails. In a word, is not the subliminal ego superior to the conscious ego? The importance of this question will be readily understood. In a recent lecture, M. Boutroux showed how it had arisen on entirely different occasions, and what consequences would be involved by an answer in the affirmative. (See also the same author’s *Science et religion*, pp. 313 et seq.)\\
于是,下面就出现了第一个假设。潜意识自我绝不比意识自我逊色;它并非纯粹自动运行;它具备辨别能力,拥有敏锐的直觉和灵活的判断力;它能够进行选择,还能做出预测。不仅如此,它比意识自我更善于预测,因为在意识自我失败的地方,它却能成功。总之,潜意识自我难道不比意识自我更优越吗?这个问题的重要性显而易见。在最近的一次讲座中,布特鲁先生阐述了这个问题是如何在完全不同的情形下产生的,以及如果给出肯定的答案会带来哪些影响。(另见同一作者的《科学与宗教》,第313页及之后的内容 ) \\

\noindent 31\\
Are we forced to give this affirmative answer by the facts I have just stated? I confess that, for my part, I should be loth to accept it. Let us, then, return to the facts, and see if they do not admit of some other explanation.\\
根据我刚刚陈述的这些事实,我们就必须给出肯定的答案吗?老实说,就我而言,我不太愿意接受这个答案。那么,让我们再回到这些事实本身,看看是否还能有其他的解释。 \\ 

\noindent 32\\
It is certain that the combinations which present themselves to the mind in a kind of sudden illumination after a somewhat prolonged period of unconscious work are generally useful and fruitful combinations, which appear to be the result of a preliminary sifting. Does it follow from this that the subliminal ego, having divined by a delicate intuition that these combinations could be useful, has formed none but these, or has it formed a great many others which were devoid of interest, and remained unconscious?\\
可以肯定的是,在经过一段时间的无意识思考后,那些以突然顿悟的形式出现在脑海中的组合,通常都是有用且富有成效的,它们似乎是经过初步筛选的结果。由此是否可以推断,潜意识自我凭借敏锐的直觉预见到这些组合可能有用,因而只构造了这些组合,还是说它也构造了许多其他毫无价值且仍处于无意识状态的组合呢? \\

\noindent 33\\
Under this second aspect, all the combinations are formed as a result of the automatic action of the subliminal ego, but those only which are interesting find their way into the field of consciousness. This, too, is most mysterious. How can we explain the fact that, of the thousand products of our unconscious activity, some are invited to cross the threshold, while others remain outside? Is it mere chance that gives them this privilege? Evidently not. For instance, of all the excitements of our senses, it is only the most intense that retain our attention, unless it has been directed upon them by other causes. More commonly the privileged unconscious phenomena, those that are capable of becoming conscious, are those which, directly or indirectly, most deeply affect our sensibility.\\
从第二个角度来看,所有的组合都是潜意识自我自动运作的结果,但只有那些有价值的组合才能进入意识领域。这同样非常神秘。我们该如何解释,在无意识活动产生的众多结果中,有些能够进入意识层面,而有些却不能呢?赋予它们这种 “特权” 的仅仅是偶然因素吗?显然不是。例如,在我们所有的感官刺激中,除非有其他因素的引导,否则只有最强烈的刺激才能引起我们的注意。更常见的情况是,那些能够进入意识层面的 “幸运” 的无意识现象,是那些直接或间接地对我们的情感产生最深刻影响的现象。 \\ 

\noindent 34\\
It may appear surprising that sensibility should be introduced in connexion with mathematical demonstrations, which, it would seem, can only interest the intellect. But not if we bear in mind the feeling of mathematical beauty, of the harmony of numbers and forms and of geometric elegance. It is a real aesthetic feeling that all true mathematicians recognize, and this is truly sensibility.\\
将情感与数学证明联系起来,这可能会让人感到惊讶,因为数学证明似乎只与智力相关。但如果我们想到数学之美,即数字与形式的和谐以及几何的优雅所带来的感觉,就不会这么觉得了。这是一种所有真正的数学家都能体会到的真实的审美感受,而这其实就是情感。 \\ 

\noindent 35\\
Now, what are the mathematical entities to which we attribute this character of beauty and elegance, which are capable of developing in us a kind of aesthetic emotion? Those whose elements are harmoniously arranged so that the mind can, without effort, take in the whole without neglecting the details. This harmony is at once a satisfaction to our aesthetic requirements, and an assistance to the mind which it supports and guides. At the same time, by setting before our eyes a well - ordered whole, it gives us a presentiment of a mathematical law. Now, as I have said above, the only mathematical facts worthy of retaining our attention and capable of being useful are those which can make us acquainted with a mathematical law. Accordingly we arrive at the following conclusion. The useful combinations are precisely the most beautiful, I mean those that can most charm that special sensibility that all mathematicians know, but of which laymen are so ignorant that they are often tempted to smile at it.\\
那么,究竟是什么样的数学内容,让我们赋予其优美和雅致的特质,还能在我们心中唤起一种审美情感呢?是那些元素和谐排列的内容,这样我们的大脑能够毫不费力地把握整体,同时又不会忽略细节。这种和谐既满足了我们的审美需求,又对思维起到支持和引导的作用。与此同时,它通过向我们展示一个井然有序的整体,让我们对数学规律有所预感。正如我之前所说,只有那些能让我们了解数学规律的数学内容,才值得我们关注且真正有用。因此,我们得出以下结论:有用的组合恰恰是最美的,我的意思是,这些组合最能吸引所有数学家都熟知的那种特殊情感,而外行人对此却一无所知,甚至常常会因此而发笑。 \\ 

\noindent 36\\
What follows, then? Of the very large number of combinations which the subliminal ego blindly forms, almost all are without interest and without utility. But, for that very reason, they are without action on the aesthetic sensibility; the consciousness will never know them. A few only are harmonious, and consequently at once useful and beautiful, and they will be capable of affecting the geometrician’s special sensibility I have been speaking of; which, once aroused, will direct our attention upon them, and will thus give them the opportunity of becoming conscious.\\
那么接下来会怎样呢?在潜意识自我盲目构造出的大量组合中,几乎绝大多数都是无趣且无用的。但也正因如此,它们不会触动审美情感,意识也永远不会察觉到它们。只有少数组合是和谐的,因而既有用又美妙,它们能够触动我一直在说的几何学家的那种特殊情感。这种情感一旦被激发,就会引导我们将注意力集中在这些组合上,从而使它们有机会进入意识层面。\\ 

\noindent 37\\
This is only a hypothesis, and yet there is an observation which tends to confirm it. When a sudden illumination invades the mathematician’s mind, it most frequently happens that it does not mislead him. But it also happens sometimes, as I have said, that it will not stand the test of verification. Well, it is to be observed almost always that this false idea, if it had been correct, would have flattered our natural instinct for mathematical elegance.\\
这只是一个假设,不过有一项观察结果似乎能证实它。当突然的顿悟出现在数学家脑海中时,通常情况下这种顿悟不会误导他。但正如我所说,有时也会出现经不起验证的情况。嗯,几乎总能发现,这些错误的想法,如果是正确的话,往往是迎合了我们对数学优雅性的本能追求。 \\ 

\noindent 38\\
Thus it is this special aesthetic sensibility that plays the part of the delicate sieve of which I spoke above, and this makes it sufficiently clear why the man who has it not will never be a real discoverer.\\
因此,正是这种特殊的审美情感,起到了我前面提到的那种精细 “筛选器” 的作用,这也十分清楚地说明了为什么没有这种审美情感的人永远成不了真正的发现者。 \\ 

\noindent 39\\
All the difficulties, however, have not disappeared. The conscious ego is strictly limited, but as regards the subliminal ego, we do not know its limitations, and that is why we are not too loth to suppose that in a brief space of time it can form more different combinations than could be comprised in the whole life of a conscient being. These limitations do exist, however. Is it conceivable that it can form all the possible combinations, whose number staggers the imagination? Nevertheless this would seem to be necessary, for if it produces only a small portion of the combinations, and that by chance, there will be very small likelihood of the right one, the one that must be selected, being found among them.\\
然而,所有的难题并没有就此消失。意识自我的能力是极为有限的,但对于潜意识自我,我们并不清楚它的局限所在,正因如此,我们不难设想,在短时间内,它能构造出比一个有思维的人一生中所能想到的还要多的不同组合。不过,这些局限确实是存在的。我们能想象它可以构造出所有可能的组合吗?其数量之多超乎想象。然而这似乎又是必要的,因为如果它只是偶然地产生一小部分组合,那么从中找到那个正确的、必须被挑选出来的组合的可能性就会非常小。 \\ 

\noindent 40\\
Perhaps we must look for the explanation in that period of preliminary conscious work which always precedes all fruitful unconscious work. If I may be permitted a crude comparison, let us represent the future elements of our combinations as something resembling Epicurus’s hooked atoms. When the mind is in complete repose these atoms are immovable; they are, so to speak, attached to the wall. This complete repose may continue indefinitely without the atoms meeting, and, consequently, without the possibility of the formation of any combination. \\
或许我们必须从所有富有成效的无意识思考之前的初步有意识思考阶段中寻找答案。如果可以做一个不太恰当的比喻,我们可以把未来组合中的元素想象成类似于伊壁鸠鲁所说的带钩原子。当大脑完全处于休息状态时,这些原子是静止不动的,可以说,它们附着在 “墙上”。这种完全的静止状态可能会无限期地持续下去,原子之间不会相遇,因此也就不可能形成任何组合。 \\ 

\noindent 41\\
On the other hand, during a period of apparent repose, but of unconscious work, some of them are detached from the wall and set in motion. They plough through space in all directions, like a swarm of gnats, for instance, or, if we prefer a more learned comparison, like the gaseous molecules in the kinetic theory of gases. Their mutual collisions may then produce new combinations.\\
另一方面,在看似休息实则进行无意识思考的阶段,其中一些原子会从 “墙上” 脱离并开始运动。它们像一群小飞虫一样,朝着各个方向在空间中穿梭,或者,如果想要一个更学术的比喻,就像气体分子运动论中的气体分子那样。它们之间相互碰撞,进而可能产生新的组合。 \\

\noindent 42\\
What is the part to be played by the preliminary conscious work? Clearly it is to liberate some of these atoms, to detach them from the wall and set them in motion. We think we have accomplished nothing, when we have stirred up the elements in a thousand different ways to try to arrange them, and have not succeeded in finding a satisfactory arrangement. But after this agitation imparted to them by our will, they do not return to their original repose, but continue to circulate freely.\\
初步有意识思考阶段起到什么作用呢?显然,它是为了解放其中一些 “原子”,使它们从 “墙上” 脱离并开始运动。当我们用成千上万种不同的方式去调动这些元素,试图对它们进行排列,却未能找到令人满意的排列方式时,我们会觉得自己一无所获。但是,在我们的意志给它们带来这种 “扰动” 之后,它们不会再回到最初的静止状态,而是会继续自由地 “运动”。 \\ 

\noindent 43\\
Now our will did not select them at random, but in pursuit of a perfectly definite aim. Those it has liberated are not, therefore, chance atoms; they are those from which we may reasonably expect the desired solution. The liberated atoms will then experience collisions, either with each other, or with the atoms that have remained stationary, which they will run against in their course. I apologize once more. My comparison is very crude, but I cannot well see how I could explain my thought in any other way.\\
我们的意志并非随意挑选这些 “原子”,而是为了追求一个明确的目标。因此,被释放的 “原子” 并非随机的;它们是我们有理由期待能得出所需解决方案的那些 “原子”。随后,被释放的 “原子” 会发生碰撞,要么是相互之间的碰撞,要么是与那些静止的 “原子” 碰撞,在运动过程中它们会撞到这些静止的 “原子” 。我再次致歉,我的比喻很粗略,但我实在想不出还能用其他什么方式来解释我的想法。\\  

\noindent 44\\
However it be, the only combinations that have any chance of being formed are those in which one at least of the elements is one of the atoms deliberately selected by our will. Now it is evidently among these that what I called just now the right combination is to be found. Perhaps there is here a means of modifying what was paradoxical in the original hypothesis.\\
无论如何,有可能形成的组合,只有那些其中至少有一个元素是我们的意志特意挑选出来的 “原子” 的组合。显然,我刚才所说的正确组合,就存在于这些组合之中。或许,这是一种修正最初假设中矛盾之处的方法。 \\ 

\noindent 45\\
Yet another observation. It never happens that unconscious work supplies ready - made the result of a lengthy calculation in which we have only to apply fixed rules. It might be supposed that the subliminal ego, purely automatic as it is, was peculiarly fitted for this kind of work, which is, in a sense, exclusively mechanical. It would seem that, by thinking overnight of the factors of a multiplication sum, we might hope to find the product ready - made for us on waking; or again, that an algebraic calculation, for instance, or a verification could be made unconsciously. Observation proves that such is by no means the case. All that we can hope from these inspirations, which are the fruits of unconscious work, is to obtain points of departure for such calculations. As for the calculations themselves, they must be made in the second period of conscious work which follows the inspiration, and in which the results of the inspiration are verified and the consequences deduced. The rules of these calculations are strict and complicated; they demand discipline, attention, will, and consequently consciousness. In the subliminal ego, on the contrary, there reigns what I would call liberty, if one could give this name to the mere absence of discipline and to disorder born of chance. Only, this very disorder permits of unexpected couplings.\\
还有一点发现。无意识思考永远不会直接给出一个冗长计算的现成结果,而这类计算我们只需应用固定的规则即可完成。人们可能会认为,潜意识自我虽然是完全自动运行的,但特别适合这类在某种意义上完全是机械性的工作。似乎我们要是在晚上思考一道乘法题的因数,就有望在醒来时发现乘积已经现成地摆在那里;又或者,诸如代数计算或验证过程也能在无意识状态下完成。然而观察表明,事实绝非如此。我们从这些作为无意识思考成果的灵感中所能期望的,仅仅是为这类计算找到一个出发点。至于计算本身,则必须在灵感之后的第二阶段有意识思考中进行,在这个阶段,要对灵感带来的结果进行验证,并推导出相应的结论。这些计算的规则既严格又复杂,需要严谨性、专注力、意志力,因而也需要意识参与。相反,在潜意识自我中,存在着一种我愿称之为 “自由” 的状态,如果可以用这个词来形容那种完全缺乏严谨性以及由偶然性导致的无序状态的话。只是,正是这种无序状态才使得意想不到的组合成为可能。 \\ 

\noindent 46\\
I will make one last remark. When I related above some personal observations, I spoke of a night of excitement, on which I worked as though in spite of myself. The cases of this are frequent, and it is not necessary that the abnormal cerebral activity should be caused by a physical stimulant, as in the case quoted. Well, it appears that, in these cases, we are ourselves assisting at our own unconscious work, which becomes partly perceptible to the overexcited consciousness, but does not on that account change its nature. We then become vaguely aware of what distinguishes the two mechanisms, or, if you will, of the methods of working of the two egos. The psychological observations I have thus succeeded in making appear to me, in their general characteristics, to confirm the views I have been enunciating. \\
我最后再说一点。当我在前面讲述一些个人经历时,曾提到过一个兴奋的夜晚,当时我不由自主地投入思考。这种情况很常见,而且异常的大脑活动并不一定像前面提到的例子那样,是由物理刺激引起的。嗯,在这些情况下,似乎我们自己见证了自身的无意识思考过程,这种过程在过度兴奋的意识中部分地变得可察觉,但并不会因此而改变其本质。然后我们会隐约意识到两种机制的区别,或者说,两种 “自我” 的工作方式的差异。就总体特征而言,我成功进行的这些心理学观察,似乎证实了我一直阐述的观点。\\  

\noindent 47\\
Truly there is great need of this, for in spite of everything they are and remain largely hypothetical. The interest of the question is so great that I do not regret having submitted them to the reader.\\
确实非常有必要这样做,因为尽管如此,这些观点在很大程度上仍然只是假设。但这个问题非常有趣,所以我并不后悔将这些观点呈现给读者。\\  

\end{document}