\documentclass{article}
\usepackage[utf8]{inputenc} % 支持UTF-8编码
\usepackage{xeCJK} % 支持中文
\usepackage{graphicx} % 引入用于插入图片的宏包
\usepackage{hyperref} % 引入超链接宏包
\usepackage{amsmath}

\usepackage{geometry}
\geometry{
  a4paper,
  left=20mm,
  right=20mm,
  top=30mm,
  bottom=30mm
}

% 设置行距为1.5倍
\usepackage{setspace}
\linespread{1.75}

\begin{document}

\title{\textbf{
GFN1000 与自然对话\\
中英双语对照翻译版
}} % 文章标题
\date{}
\maketitle % 生成标题

\setcounter{secnumdepth}{0} % 禁止章节编号,但仍添加到目录
\tableofcontents
\newpage
\section{Text 6 from Silent Spring/ \textit{Rachel Carson}}
\begin{center}
    CHAPTER 6\\
    EARTH’S GREEN MANTLE 地球的绿色披风\\
\end{center}

\noindent 1\\
Water, soil, and the earth’s green mantle of plants make up the world that supports the animal life of the earth. Although modern man seldom remembers the fact, he could not exist without the plants that harness the sun’s energy and manufacture the basic foodstuffs he depends upon for life. Our attitude toward plants is a singularly narrow one. If we see any immediate utility in a plant we foster it. If for any reason we find its presence undesirable or merely a matter of indifference, we may condemn it to destruction forthwith. Besides the various plants that are poisonous to man or his livestock, or crowd out food plants, many are marked for destruction merely because, according to our narrow view, they happen to be in the wrong place at the wrong time. Many others are destroyed merely because they happen to be associates of the unwanted plants.\\
水、土壤以及地球上植物构成的绿色 “披风”,共同组成了维系地球上动物生命的世界。尽管现代人很少想起这一点,但如果没有植物利用太阳能并生产出人类赖以生存的基本食物,人类就无法生存。我们对待植物的态度极其狭隘。如果发现某种植物有直接用途,我们就会培育它。如果出于某种原因,我们觉得某种植物的存在不合心意,或者对它毫不在意,我们可能会立刻将其铲除。除了那些对人类或牲畜有毒,或者会排挤食用植物的各类植物外,还有许多植物仅仅因为在我们狭隘的认知中,它们出现在了不该出现的时间和地点,就被列入了清除名单。还有许多其他植物被毁掉,仅仅是因为它们碰巧与那些不想要的植物共生在一起 。\\

\noindent 2\\
The earth’s vegetation is part of a web of life in which there are intimate and essential relations between plants and the earth, between plants and other plants, between plants and animals. Sometimes we have no choice but to disturb these relationships, but we should do so thoughtfully, with full awareness that what we do may have consequences remote in time and place. But no such humility marks the booming “weed killer” business of the present day, in which soaring sales and expanding uses mark the production of plant-killing chemicals.\\
地球上的植被是生命网络的一部分,在这个网络中,植物与土壤之间、植物与其他植物之间、植物与动物之间存在着紧密且至关重要的联系。有时,我们别无选择,只能干扰这些关系,但我们应该谨慎行事,充分意识到我们的所作所为可能会在时间和空间上产生深远的影响。然而,如今蓬勃发展的 “除草剂” 产业却毫无这种敬畏之心,在这个产业中,销售额的飙升和使用范围的扩大,成为了生产植物杀伤性化学药剂的显著特征。 \\

\noindent 3\\
One of the most tragic examples of our unthinking bludgeoning of the landscape is to be seen in the sagebrush lands of the West, where a vast campaign is on to destroy the sage and to substitute grasslands. If ever an enterprise needed to be illuminated with a sense of the history and meaning of the landscape, it is this. For here the natural landscape is eloquent of the interplay of forces that have created it. It is spread before us like the pages of an open book in which we can read why the land is what it is, and why we should preserve its integrity. But the pages lie unread.\\
我们不假思索地破坏自然景观,其中一个最为惨痛的例子,就发生在西部的山艾树生长地区,那里正在开展一场大规模铲除山艾树、改种草地的行动。如果说有什么活动需要从景观的历史和意义层面进行考量,那就是这个了。因为这里的自然景观充分展现了塑造它的各种力量之间的相互作用。它就像一本摊开的书呈现在我们面前,从中我们可以了解这片土地为何是现在的模样,以及我们为何应该保护它的完整。然而,这些 “书页” 却无人翻阅。 \\

\noindent 4\\
The land of the sage is the land of the high western plains and the lower slopes of the mountains that rise above them, a land born of the great uplift of the Rocky Mountain system many millions of years ago. It is a place of harsh extremes of climate: of long winters when blizzards drive down from the mountains and snow lies deep on the plains, of summers whose heat is relieved by only scanty rains, with drought biting deep into the soil, and drying winds stealing moisture from leaf and stem.\\\\
山艾树生长的地区位于西部高平原,以及平原之上山脉的低坡处,这片土地形成于数百万年前落基山脉的大规模隆起。这里气候条件极为恶劣:冬季漫长,暴风雪从山上呼啸而下,平原上积雪深厚;夏季酷热,只有少量降雨稍作缓解,干旱深入土壤,干燥的风还会带走植物茎叶中的水分。 \\\\

\noindent 5\\
As the landscape evolved, there must have been a long period of trial and error in which plants attempted the colonization of this high and windswept land. One after another must have failed. At last one group of plants evolved which combined all the qualities needed to survive. The sage—low-growing and shrubby—could hold its place on the mountain slopes and on the plains, and within its small gray leaves it could hold moisture enough to defy the thieving winds. It was no accident, but rather the result of long ages of experimentation by nature, that the great plains of the West became the land of the sage.\\\\
随着地貌的演变,必然有一段漫长的反复试错期,期间各种植物试图在这片高耸且多风的土地上扎根生长。一种又一种植物尝试后都失败了。最终,有一类植物进化出了生存所需的所有特性。山艾树植株低矮且呈灌木状,能够在山坡和平原上扎根生长,其小小的灰绿色叶片能够储存足够的水分,抵御干燥的风。西部大平原成为山艾树的生长之地并非偶然,而是大自然历经漫长岁月 “实验” 的结果。 \\\\

\noindent 6\\
Along with the plants, animal life, too, was evolving in harmony with the searching requirements of the land. In time there were two as perfectly adjusted to their habitat as the sage. One was a mammal, the fleet and graceful pronghorn antelope. The other was a bird, the sage grouse—the “cock of the plains” of Lewis and Clark.\\\\
与植物一同,动物的生命也在与这片土地严苛的生存条件相适应,不断和谐地进化着。 最终,有两种动物像山艾树一样完美地适应了它们的栖息地。 一种是哺乳动物,敏捷优雅的叉角羚。 另一种是鸟类,艾草松鸡——即刘易斯和克拉克笔下的 “平原之雄”。\\\\

\noindent 7\\
The sage and the grouse seem made for each other. The original range of the bird coincided with the range of the sage, and as the sagelands have been reduced, so the populations of grouse have dwindled. The sage is all things to these birds of the plains. The low sage of the foothill ranges shelters their nests and their young; the denser growths are loafing and roosting areas; at all times the sage provides the staple food of the grouse. Yet it is a two - way relationship. The spectacular courtship displays of the cocks help loosen the soil beneath and around the sage, aiding invasion by grasses which grow in the shelter of sagebrush.\\
山艾树和艾草松鸡似乎天生一对。这种鸟最初的栖息地范围与山艾树的生长范围重合,随着山艾树生长地面积减少,艾草松鸡的数量也随之减少。对于这些平原上的鸟儿来说,山艾树至关重要。山麓地带低矮的山艾树为它们的巢穴和幼鸟提供庇护;较为茂密的山艾树区域则是它们休息和栖息的地方;山艾树在任何时候都是艾草松鸡的主要食物来源。然而,这是一种双向关系。雄鸟壮观的求偶表演有助于松动山艾树下方和周围的土壤,这有利于在山艾树庇护下生长的草类植物的繁衍。 \\

\noindent 8\\
The antelope, too, have adjusted their lives to the sage. They are primarily animals of the plains, and in winter when the first snows come those that have summered in the mountains move down to the lower elevations. There the sage provides the food that tides them over the winter. Where all other plants have shed their leaves, the sage remains evergreen, the gray-green leaves—bitter, aromatic, rich in proteins, fats, and needed minerals—clinging to the stems of the dense and shrubby plants. Though the snows pile up, the tops of the sage remain exposed, or can be reached by the sharp, pawing hoofs of the antelope. Then grouse feed on them too, finding them on bare and windswept ledges or following the antelope to feed where they have scratched away the snow.\\
叉角羚也使自己的生活适应了山艾树。它们主要生活在平原上,冬季初雪降临之时,那些在山区度过夏天的叉角羚会迁徙到海拔较低的地方。在那里,山艾树为它们提供了过冬的食物。当其他植物都已落叶,山艾树却依然常绿,灰绿色的叶子——味道苦涩、散发着香气,富含蛋白质、脂肪和必需的矿物质——紧紧附着在茂密的灌木茎上。尽管积雪堆积,山艾树的顶部依然外露,或者叉角羚可以用锋利的蹄子刨开积雪够到。艾草松鸡也会以山艾树为食,它们会在光秃秃、狂风肆虐的岩架上寻找山艾树,或者跟着叉角羚,到它们刨开积雪的地方觅食。 \\

\noindent 9\\
And other life looks to the sage. Mule deer often feed on it. Sage may mean survival for winter-grazing livestock. Sheep graze many winter ranges where the big sagebrush forms almost pure stands. For half the year it is their principal forage, a plant of higher energy value than even alfalfa hay.\\
其他生物也依赖着山艾树。长耳鹿常以山艾树为食。对于冬季放牧的牲畜而言,山艾树可能意味着生存。在许多绵羊冬季放牧的牧场,大片的山艾树几乎形成了纯林。在半年的时间里,山艾树是它们的主要草料,这种植物的能量价值甚至比苜蓿干草还要高。 \\

\noindent 10\\
The bitter upland plains, the purple wastes of sage, the wild, swift antelope, and the grouse are then a natural system in perfect balance. Are? The verb must be changed—at least in those already vast and growing areas where man is attempting to improve on nature’s way. In the name of progress the land management agencies have set about to satisfy the insatiable demands of the cattlemen for more grazing land. By this they mean grassland—grass without sage. So in a land which nature found suited to grass growing mixed with and under the shelter of sage, it is now proposed to eliminate the sage and create unbroken grassland. Few seem to have asked whether grasslands are a stable and desirable goal in this region. Certainly nature’s own answer was otherwise. The annual precipitation in this land where the rains seldom fall is not enough to support good sod-forming grass; it favors rather the perennial bunch grass that grows in the shelter of the sage.\\\\
苦涩的高原平原、紫色的山艾树荒原、敏捷的野生叉角羚和艾草松鸡,曾构成一个完美平衡的自然系统。是这样吗?这个 “是” 得改一改了 —— 至少在那些广大且不断扩大的区域里是这样,在这些地方人类试图改变自然的运作方式。以进步之名,土地管理机构着手满足牧场主对更多牧场的无尽需求。他们所说的牧场,指的是没有山艾树的草地。所以,在这片自然认为适合草本植物与山艾树混生,或在山艾树庇护下生长的土地上,如今却有人提议铲除山艾树,打造连片的草地。似乎很少有人思考,在这个地区,草地是否是一个稳定且理想的目标。显然,大自然给出的答案并非如此。在这片很少降雨的土地上,年降水量不足以维持能形成良好草皮的草类生长;它更适合在山艾树庇护下生长的多年生丛生草。\\

\noindent 11\\
Yet the program of sage eradication has been under way for a number of years. Several government agencies are active in it; industry has joined with enthusiasm to promote and encourage an enterprise which creates expanded markets not only for grass seed but for a large assortment of machines for cutting and plowing and seeding. The newest addition to the weapons is the use of chemical sprays. Now millions of acres of sagebrush lands are sprayed each year.\\
然而,铲除山艾树的计划已经推行了数年。几家政府机构积极参与其中;相关产业也热情高涨地推动并鼓励这一事业,因为它不仅为草种开辟了更广阔的市场,也为大量用于割草、翻耕和播种的机械设备创造了市场。在铲除山艾树的 “武器库” 中,最新加入的是化学药剂喷洒。如今,每年有数百万英亩的山艾树生长地被喷洒药剂。 \\

\noindent 12\\
What are the results? The eventual effects of eliminating sage and seeding with grass are largely conjectural. Men of long experience with the ways of the land say that in this country there is better growth of grass between and under the sage than can possibly be had in pure stands, once the moisture-holding sage is gone.\\
结果如何呢?铲除山艾树并种草的最终影响在很大程度上还只是猜测。对土地特性有长期了解的人说,在这个地区,山艾树之间和下方的草长得比单纯的草地更好,而一旦具有保水能力的山艾树消失,情况就不同了。 \\

\noindent 13\\
But even if the program succeeds in its immediate objective, it is clear that the whole closely knit fabric of life has been ripped apart. The antelope and the grouse will disappear along with the sage. The deer will suffer, too, and the land will be poorer for the destruction of the wild things that belong to it. Even the livestock which are the intended beneficiaries will suffer; no amount of lush green grass in summer can help the sheep starving in the winter storms for lack of the sage and bitterbrush and other wild vegetation of the plains.\\
然而,即使这个计划在短期内达成了目标,显然整个紧密相连的生命网络已被撕裂。叉角羚和艾草松鸡会随着山艾树一同消失。鹿群也会遭殃,而且这片土地会因为原本栖息于此的野生物种遭到破坏而变得贫瘠。就连原本计划中的受益方——牲畜也会受到影响;夏季再茂盛的绿草,也无法帮助那些在冬季暴风雪中,因缺少山艾树、苦灌木和其他平原野生植被而挨饿的羊群。 \\

\noindent 14\\
These are the first and obvious effects. The second is of a kind that is always associated with the shotgun approach to nature: the spraying also eliminates a great many plants that were not its intended target. Justice William O. Douglas, in his recent book *My Wilderness: East to Katahdin*, has told of an appalling example of ecological destruction wrought by the United States Forest Service in the Bridger National Forest in Wyoming. Some 10,000 acres of sagelands were sprayed by the Service, yielding to pressure of cattlemen for more grasslands. The sage was killed, as intended. But so was the green, life giving ribbon of willows that traced its way across these plains, following the meandering streams. Moose had lived in these willow thickets, for willow is to the moose what sage is to the antelope. Beaver had lived there, too, feeding on the willows, felling them and making a strong dam across the tiny stream. Through the labor of the beavers, a lake backed up. Trout in the mountain streams seldom were more than six inches long; in the lake they thrived so prodigiously that many grew to five pounds. Waterfowl were attracted to the lake, also. Merely because of the presence of the willows and the beavers that depended on them, the region was an attractive recreational area with excellent fishing and hunting.\\
这些是首要且明显的影响。其次是一种总是与粗暴对待自然的方式相关联的后果:喷洒药剂还会除掉大量并非目标的植物。大法官威廉·O·道格拉斯在他最近的著作《我的荒野:向东到卡塔丁》中,讲述了美国林业局在怀俄明州布里杰国家森林造成生态破坏的一个惊人案例。林业局迫于牧场主对更多牧场的压力,对约1万英亩的山艾树生长地进行了喷洒作业。山艾树如预期那样被除掉了。但沿着蜿蜒溪流穿过这片平原的、充满生机的绿色柳树林带也未能幸免。驼鹿曾栖息在这些柳树林中,柳树之于驼鹿,就如同山艾树之于叉角羚。海狸也曾生活在那里,以柳树为食,砍伐柳树并在小溪上筑造坚固的水坝。由于海狸的劳作,形成了一个堰塞湖。山间溪流中的鳟鱼很少有超过6英寸长的;而在这个湖里,它们大量繁衍,许多长到了5磅重。水鸟也被吸引到这个湖边。仅仅因为柳树和依赖柳树生存的海狸的存在,这个地区曾是一个颇具吸引力的休闲区,有着绝佳的钓鱼和狩猎条件。 \\

\noindent 15\\
But with the “improvement” instituted by the Forest Service, the willows went the way of the sagebrush, killed by the same impartial spray. When Justice Douglas visited the area in 1959, the year of the spraying, he was shocked to see the shriveled and dying willows—the “vast, incredible damage.” What would become of the moose? Of the beavers and the little world they had constructed? A year later he returned to read the answers in the devastated landscape. The moose were gone and so were the beaver. Their principal dam had gone out for want of attention by its skilled architects, and the lake had drained away. None of the large trout were left. None could live in the tiny creek that remained, threading its way through a bare, hot land where no shade remained. The living world was shattered.\\
然而,随着林业局推行的 “改良” 措施实施,柳树也像山艾树一样,被同样不加区分的喷洒药剂给杀死了。1959年,也就是进行喷洒作业的那一年,道格拉斯大法官访问了该地区,看到那些枯萎濒死的柳树,他大为震惊,称这是 “巨大且令人难以置信的破坏” 。驼鹿会怎么样呢?海狸以及它们所构建的小世界又会怎样呢?一年后,他回到这片遭到破坏的地方寻找答案。驼鹿不见了,海狸也消失了。由于它们技艺精湛的 “建筑师”(海狸 )不复存在,主要的水坝也毁坏了,湖水随之干涸。大型的鳟鱼一条也没剩下。没有鱼能在仅存的小溪中生存,这条小溪蜿蜒流过一片光秃秃、酷热且没有任何遮荫的土地。生物世界被摧毁了。\\

\noindent 16\\
Besides the more than four million acres of rangelands sprayed each year, tremendous areas of other types of land are also potential or actual recipients of chemical treatments for weed control. For example, an area larger than all of New England—some 50 million acres—is under management by utility corporations and much of it is routinely treated for “brush control.” In the Southwest an estimated 75 million acres of mesquite lands require management by some means, and chemical spraying is the method most actively pushed. An unknown but very large acreage of timber - producing lands is now aerially sprayed in order to “weed out” the hardwoods from the more spray - resistant conifers. Treatment of agricultural lands with herbicides doubled in the decade following 1949, totaling 53 million acres in 1959. And the combined acreage of private lawns, parks, and golf courses now being treated must reach an astronomical figure.\\
除了每年有超过400万英亩的牧场被喷洒药剂外,大量其他类型的土地也可能或已经接受了化学除草处理。例如,一片比整个新英格兰地区还要大(约5000万英亩 )的土地由公用事业公司管理,其中大部分都按惯例进行 “灌木控制” 处理。在西南部,估计有7500万英亩的牧豆树生长地需要采取某种方式进行管理,而化学药剂喷洒是最积极推行的方法。为了从更具抗药性的针叶树中 “清除” 阔叶树,目前有数量不明但面积巨大的林地正在进行空中喷洒作业。1949年后的十年间,使用除草剂处理的农田面积翻了一番,到1959年达到了5300万英亩。目前,正在接受处理的私人草坪、公园和高尔夫球场的总面积肯定是一个惊人的数字。 \\

\noindent 17\\
The chemical weed killers are a bright new toy. They work in a spectacular way; they give a giddy sense of power over nature to those who wield them, and as for the long-range and less obvious effects—these are easily brushed aside as the baseless imaginings of pessimists. The “agricultural engineers” speak blithely of “chemical plowing” in a world that is urged to beat its plowshares into spray guns. The town fathers of a thousand communities lend willing ears to the chemical salesman and the eager contractors who will rid the roadsides of “brush”—for a price. It is cheaper than mowing, is the cry. So, perhaps, it appears in the neat rows of figures in the official books; but were the true costs entered, the costs not only in dollars but in the many equally valid debits we shall presently consider, the wholesale broadcasting of chemicals would be seen to be more costly in dollars as well as infinitely damaging to the long-range health of the landscape and to all the varied interests that depend on it.\\
化学除草剂是一个新奇的玩意儿。它们的效果惊人,让使用它们的人产生一种掌控自然的飘飘然之感。而对于其长期的、不那么明显的影响,人们却轻易地将其视为悲观主义者毫无根据的臆想而置之不理。在这个被鼓动着将犁铧改造成喷枪的世界里,“农业工程师” 们轻松地谈论着 “化学耕作”。无数社区的官员们欣然倾听化学品推销员和心急的承包商们的推销——他们能以一定的价格清除道路两旁的 “灌木”。人们呼喊着说这比割草便宜。也许,在官方账本整齐的数字栏里看起来是这样;但如果算上真正的成本,不仅是金钱方面的成本,还包括我们接下来要考虑的许多同样重要的 “损耗”,就会发现大规模喷洒化学品不仅在经济上代价更高,而且对自然景观的长期健康以及依赖它的各种利益也会造成极大的损害。 \\

\noindent 18\\
Take, for instance, that commodity prized by every chamber of commerce throughout the land—the good will of vacationing tourists. There is a steadily growing chorus of outraged protest about the disfigurement of once beautiful roadsides by chemical sprays, which substitute a sere expanse of brown, withered vegetation for the beauty of fern and wildflower, of native shrubs adorned with blossom or berry. “We are making a dirty, brown, dying-looking mess along the sides of our roads,” a New England woman wrote angrily to her newspaper. “This is not what the tourists expect, with all the money we are spending advertising the beautiful scenery.”\\
以全国各地的商会都珍视的一种 “商品” 为例,那就是度假游客的好感。曾经美丽的道路两旁被化学药剂喷洒得面目全非,大片枯黄、凋零的植被取代了原本有蕨类植物、野花,以及点缀着花朵或浆果的本土灌木所构成的美景,对此,愤怒的抗议声日益高涨。“我们把道路两旁弄得又脏又黄,一副死气沉沉的样子,” 一位新英格兰地区的女士愤怒地写信给当地报纸,“我们花了那么多钱宣传这里的美丽风景,可游客们看到的却不是这样。”\\

\noindent 19\\
In the summer of 1960 conservationists from many states converged on a peaceful Maine island to witness its presentation to the National Audubon Society by its owner, Millicent Todd Bingham. The focus that day was on the preservation of the natural landscape and of the intricate web of life whose interwoven strands lead from microbes to man. But in the background of all the conversations among the visitors to the island was indignation at the despoiling of the roads they had traveled. Once it had been a joy to follow those roads through the evergreen forests, roads lined with bayberry and sweet fern, alder and huckleberry. Now all was brown desolation. One of the conservationists wrote of that August pilgrimage to a Maine island: “I returned... angry at the desecration of the Maine roadsides. Where, in previous years, the highways were bordered with wildflowers and attractive shrubs, there were only the scars of dead vegetation for mile after mile.... As an economic proposition, can Maine afford the loss of tourist goodwill that such sights induce?”\\
1960年夏天,来自多个州的环保主义者齐聚在缅因州一座宁静的小岛上,见证岛主米利森特·托德·宾厄姆将该岛捐赠给全国奥杜邦学会。当天的焦点是保护自然景观,以及保护从微生物到人类相互交织的复杂生命网络。但在所有登岛游客的交谈中,都隐藏着对他们沿途所经道路遭到破坏的愤慨。曾经,沿着这些穿过常绿森林的道路前行是一件令人愉悦的事,道路两旁长满了蜡杨梅、香蕨木、桤木和越橘。而如今,映入眼帘的只有一片枯黄荒凉的景象。一位环保主义者在描述那年8月前往缅因州这座小岛的 “朝圣之旅” 时写道:“我回来后…… 对缅因州路边遭到的破坏感到愤怒。前些年,高速公路两旁野花盛开,灌木迷人,而现在,绵延数英里只有枯死植被留下的痕迹…… 从经济角度考量,缅因州能承受得起因这种景象而导致的游客好感流失吗” \\

\noindent 20\\
Maine roadsides are merely one example, though a particularly sad one for those of us who have a deep love for the beauty of that state, of the senseless destruction that is going on in the name of roadside brush control throughout the nation.\\
缅因州的路边情况仅仅是一个例子,尽管对于我们这些深深热爱该州美景的人来说,这是一个尤为痛心的例子,它体现了在全国范围内,以控制路边灌木为名而进行的毫无意义的破坏行为。\\ 

\noindent 21\\
Botanists at the Connecticut Arboretum declare that the elimination of beautiful native shrubs and wildflowers has reached the proportions of a “roadside crisis,” Azaleas, mountain laurel, blueberries, huckleberries, viburnums, dogwood, bayberry, sweet fern, low shadbrush, winterberry, chokecherry, and wild plum are dying before the chemical barrage. So are the daisies, black-eyed Susans, Queen Anne’s lace, goldenrods, and fall asters which lend grace and beauty to the landscape.\\
康涅狄格州植物园的植物学家宣称,对美丽本土灌木和野花的铲除已达到 “路边危机” 的程度。杜鹃花、山月桂、蓝莓、越橘、荚蒾、山茱萸、蜡杨梅、香蕨木、矮唐棣、冬浆果、稠李和野李子,在化学药剂的 “攻击” 下逐渐凋零。雏菊、黑心菊、胡萝卜花、一枝黄花和秋紫菀这些为景色增添优雅和美丽的花卉也同样如此。 \\

\noindent 22\\
The spraying is not only improperly planned but studded with abuses such as these. In a southern New England town one contractor finished his work with some chemical remaining in his tank. He discharged this along woodland roadsides where no spraying had been authorized. As a result the community lost the blue and golden beauty of its autumn roads, where asters and goldenrod would have made a display worth traveling far to see. In another New England community a contractor changed the state specifications for town spraying without the knowledge of the highway department and sprayed roadside vegetation to a height of eight feet instead of the specified maximum of four feet, leaving a broad, disfiguring, brown swath. In a Massachusetts community the town officials purchased a weed killer from a zealous chemical salesman, unaware that it contained arsenic. One result of the subsequent roadside spraying was the death of a dozen cows from arsenic poisoning.\\\\
这种喷洒作业不仅规划不当,还充斥着诸如此类的滥用情况。在新英格兰南部的一个小镇,一位承包商完成工作后,罐子里还残留了一些化学药剂。他在未经授权喷洒的林地路边将这些药剂排放了出去。结果,这个社区失去了秋日道路上蓝黄交织的美景,原本那里盛开的紫菀和一枝黄花,美得值得人们远道而来观赏。在新英格兰的另一个社区,一位承包商在公路部门不知情的情况下,更改了城镇喷洒作业的州定规范,将路边植被的喷洒高度喷到了8英尺,而规定的最大高度是4英尺,留下了一道又宽又难看的枯黄痕迹。在马萨诸塞州的一个社区,镇政府官员从一位热心的化学品推销员那里购买了一种除草剂,却不知道其中含有砷。随后进行的路边喷洒作业导致的一个后果是,12头牛因砷中毒死亡。 \\

\noindent 23\\
Trees within the Connecticut Arboretum Natural Area were seriously injured when the town of Waterford sprayed the roadsides with chemical weed killers in 1957. Even large trees not directly sprayed were affected. The leaves of the oaks began to curl and turn brown, although it was the season for spring growth. Then new shoots began to be put forth and grew with abnormal rapidity, giving a weeping appearance to the trees. Two seasons later, large branches on these trees had died, others were without leaves, and the deformed, weeping effect of whole trees persisted.\\
1957 年,沃特福德镇在路边喷洒化学除草剂,这致使康涅狄格州植物园自然区内的树木遭到严重损害。就连没有被直接喷洒到的大树也受到了影响。橡树的叶子开始卷曲并变成褐色,尽管当时正值春季生长季节。随后,新的嫩枝开始抽出,并且异常快速地生长,使树木呈现出一种 “垂泪” 的样子。两个季节后,这些树上的粗大枝干开始枯死,其他枝干则光秃秃没有叶子,整棵树扭曲、 “垂泪” 的不良状况一直持续着。 \\

\noindent 24\\
I know well a stretch of road where nature’s own landscaping has provided a border of alder, viburnum, sweet fern, and juniper with seasonally changing accents of bright flowers, or of fruits hanging in jeweled clusters in the fall. The road had no heavy load of traffic to support; there were few sharp curves or intersections where brush could obstruct the driver’s vision. But the sprayers took over and the miles along that road became something to be traversed quickly, a sight to be endured with one’s mind closed to thoughts of the sterile and hideous world we are letting our technicians make. But here and there authority had somehow faltered and by an unaccountable oversight there were oases of beauty in the midst of austere and regimented control—oases that made the desecration of the greater part of the road the more unbearable. In such places my spirit lifted to the sight of the drifts of white clover or the clouds of purple vetch with here and there the flaming cup of a wood lily.\\
我熟知一条公路,大自然在这里造就了天然的景观,桤木、荚蒾、香蕨木和刺柏构成道路的边缘,四季变换,繁花点缀其间,到了秋天,成串的果实如宝石般悬挂着 。这条公路的交通流量并不大,几乎没有会让灌木阻挡司机视线的急转弯或交叉路口。然而,喷洒工人来了,此后,沿着这条路行驶的数英里路程变得让人只想匆匆掠过,这景象让人难以忍受,不禁让人想到我们正任由技术人员打造一个毫无生机且丑陋的世界。但偶尔,管控不知为何有所松懈,由于莫名其妙的疏忽,在严苛规整的管控中出现了美丽的 “绿洲” ——这些 “绿洲” 让公路上大部分区域遭受的破坏更让人难以忍受。在这些地方,看到大片的白三叶草,如云般的紫花苕,以及随处可见像燃烧的酒杯般的林芝花,我的心情也随之振奋起来。 \\

\noindent 25\\
Such plants are “weeds” only to those who make a business of selling and applying chemicals. In a volume of Proceedings of one of the weed - control conferences that are now regular institutions, I once read an extraordinary statement of a weed killer’s philosophy. The author defended the killing of good plants “simply because they are in bad company.” Those who complain about killing wildflowers along roadsides reminded him, he said, of antivivisectionists “to whom, if one were to judge by their actions, the life of a stray dog is more sacred than the lives of children.”\\
只有对于那些以销售和使用化学药剂为营生的人来说,这些植物才是 “杂草” 。在如今已成为常规活动的除草会议的某一卷会议记录中,我曾读到过一篇关于除草剂使用理念的惊人言论。作者为铲除优良植物的行为辩护,称 “仅仅是因为它们与杂草长在了一起” 。他说,那些抱怨路边野花被铲除的人,让他联想到了反对活体解剖者,“从他们的行为判断,对这些人来说,一只流浪狗的生命比孩子们的生命还要神圣”。 \\

\noindent 26\\
To the author of this paper, many of us would unquestionably be suspect, convicted of some deep perversion of character because we prefer the sight of the vetch and the clover and the wood lily in all their delicate and transient beauty to that of roadsides scorched as by fire, the shrubs brown and brittle, the bracken that once lifted high its proud lacework now withered and drooping. We would seem deplorably weak that we can tolerate the sight of such “weeds,” that we do not rejoice in their eradication, that we are not filled with exultation that man has once more triumphed over miscreant nature.\\
对于写下上述言论的作者来说,我们中的许多人无疑会受到怀疑,会被判定为有着某种严重的性格扭曲,因为相较于那些被化学药剂摧残得如同被火烧过的路边景象——灌木枯黄脆弱,曾经高高舒展着精美叶片的欧洲蕨如今枯萎下垂,我们更钟情于紫花苕、三叶草和林芝花那精致而转瞬即逝的美丽。在他看来,我们能够容忍这些 “杂草” 的存在,不会为它们被铲除而感到欣喜,也不会为人类再次战胜 “叛逆” 的大自然而欢呼雀跃,这显得我们无比懦弱。 \\

\noindent 27\\
Justice Douglas tells of attending a meeting of federal field men who were discussing protests by citizens against plans for the spraying of sagebrush that I mentioned earlier in this chapter. These men considered it hilariously funny that an old lady had opposed the plan because the wildflowers would be destroyed. “Yet, was not her right to search out a banded cup or a tiger lily as inalienable as the right of stockmen to search out grass or of a lumberman to claim a tree?” asks this humane and perceptive jurist. “The esthetic values of the wilderness are as much our inheritance as the veins of copper and gold in our hills and the forests in our mountains.”\\
道格拉斯大法官讲述了他参加的一次联邦野外工作人员会议,会上他们讨论了民众对我在本章前面提到的喷洒山艾树计划的抗议。这些工作人员觉得一位老妇人因野花会被毁掉而反对该计划的事十分滑稽可笑。这位善良且有洞察力的法学家反问道:“然而,她寻找斑叶兰或卷丹的权利,难道不像牧民寻找青草、伐木工认领树木的权利那样不可剥夺吗 ?” “荒野的美学价值,如同山中的铜金矿脉和山林一样,都是我们的遗产。” \\

\noindent 28\\
There is of course more to the wish to preserve our roadside vegetation than even such esthetic considerations. In the economy of nature the natural vegetation has its essential place. Hedgerows along country roads and bordering fields provide food, cover, and nesting areas for birds and homes for many small animals. Of some 70 species of shrubs and vines that are typical roadside species in the eastern states alone, about 65 are important to wildlife as food.\\
当然,我们想要保护路边植被,其原因甚至不止于这些美学方面的考量。在自然生态体系中,天然植被有着不可或缺的地位。乡间道路旁以及田野边缘的树篱,为鸟类提供了食物、栖息处和筑巢地,也为许多小动物提供了家园。仅在东部各州,大约70种典型的路边灌木和藤本植物中,就有65种对野生动物而言是重要的食物来源。 \\ 

\noindent 29\\
Such vegetation is also the habitat of wild bees and other pollinating insects. Man is more dependent on these wild pollinators than he usually realizes. Even the farmer himself seldom understands the value of wild bees and often participates in the very measures that rob him of their services. Some agricultural crops and many wild plants are partly or wholly dependent on the services of the native pollinating insects. Several hundred species of wild bees take part in the pollination of cultivated crops—100 species visiting the flowers of alfalfa alone. Without insect pollination, most of the soil - holding and soil - enriching plants of uncultivated areas would die out, with far - reaching consequences to the ecology of the whole region. Many herbs, shrubs, and trees of forests and range depend on native insects for their reproduction; without these plants many wild animals and range stock would find little food. Now clean cultivation and the chemical destruction of hedgerows and weeds are eliminating the last sanctuaries of these pollinating insects and breaking the threads that bind life to life.\\
这类植被也是野生蜜蜂和其他传粉昆虫的栖息地。人类对这些野生传粉者的依赖程度,比通常所意识到的要高得多。就连农民自己也很少了解野生蜜蜂的价值,还常常参与一些剥夺它们传粉作用的活动。一些农作物和许多野生植物,部分或完全依赖本土传粉昆虫的帮助。数百种野生蜜蜂参与农作物的授粉过程,仅苜蓿花就吸引100种野生蜜蜂前来。如果没有昆虫授粉,未开垦地区的大多数固土和肥土植物将会灭绝,这会给整个地区的生态带来深远的影响。森林和牧场中的许多草本植物、灌木和树木都依赖本土昆虫进行繁殖;没有了这些植物,许多野生动物和牧场牲畜将很难找到食物。如今,精耕细作以及用化学药剂铲除树篱和杂草,正在摧毁这些传粉昆虫最后的栖息地,切断维系生命的纽带。\\  

\noindent 30\\
These insects, so essential to our agriculture and indeed to our landscape as we know it, deserve something better from us than the senseless destruction of their habitat. Honeybees and wild bees depend heavily on such “weeds” as goldenrod, mustard, and dandelions for pollen that serves as the food of their young. Vetch furnishes essential spring forage for bees before the alfalfa is in bloom, tiding them over this early season so that they are ready to pollinate the alfalfa. In the fall they depend on goldenrod at a season when no other food is available, to stock up for the winter. By the precise and delicate timing that is nature’s own, the emergence of one species of wild bees takes place on the very day of the opening of the willow blossoms. There is no dearth of men who understand these things, but these are not the men who order the wholesale drenching of the landscape with chemicals.\\
这些昆虫对我们的农业至关重要,实际上对我们熟知的自然景观也同样重要,它们不应遭受我们对其栖息地的无端破坏,而应得到更好的对待。蜜蜂和野生蜂在很大程度上依赖一枝黄花、芥菜和蒲公英等 “杂草” 提供花粉,来喂养它们的幼虫。在苜蓿开花之前,紫花苕为蜜蜂提供了重要的春季食物,帮助它们度过这段早期时节,以便能为苜蓿授粉。到了秋季,在没有其他食物来源的时候,它们依靠一枝黄花来储备过冬的食物。凭借大自然精准而微妙的时序安排,有一种野生蜂会在柳树开花的当天破茧而出。了解这些知识的人并不少,但那些下令大规模用化学药剂喷洒自然景观的人,却对这些一无所知。 \\

\noindent 31\\ 
And where are the men who supposedly understand the value of proper habitat for the preservation of wildlife? Too many of them are to be found defending herbicides as “harmless” to wildlife because they are thought to be less toxic than insecticides. Therefore, it is said, no harm is done. But as the herbicides rain down on forest and field, on marsh and rangeland, they are bringing about marked changes and even permanent destruction of wildlife habitat. To destroy the homes and the food of wildlife is perhaps worse in the long run than direct killing.\\
那些理应懂得合适的栖息地对保护野生动物具有重要价值的人都去哪儿了呢?太多这样的人声称除草剂对野生动物 “无害”,理由是他们认为除草剂的毒性比杀虫剂小。因此,他们说,不会造成危害。但是,当除草剂如雨点般落在森林、田野、沼泽和牧场时,它们正在对野生动物的栖息地造成显著改变,甚至是永久性的破坏。从长远来看,破坏野生动物的家园和食物,或许比直接捕杀它们更为恶劣。\\  

\noindent 32\\
The irony of this all-out chemical assault on roadsides and utility rights-of-way is twofold. It is perpetuating the problem it seeks to correct, for as experience has clearly shown, the blanket application of herbicides does not permanently control roadside “brush” and the spraying has to be repeated year after year. And as a further irony, we persist in doing this despite the fact that a perfectly sound method of selective spraying is known, which can achieve long-term vegetation control and eliminate repeated spraying in most types of vegetation.\\
这种对路边和公用设施用地进行全面化学药剂喷洒的做法存在两方面的讽刺之处。它使得原本试图解决的问题一直持续存在,因为经验已明确表明,大面积使用除草剂并不能永久性地控制路边的 “灌木”,而且这种喷洒作业不得不年复一年地重复进行。更具讽刺意味的是,尽管我们知道有一种十分可靠的选择性喷洒方法,它能够实现对植被的长期控制,并且在大多数植被类型中无需反复喷洒,但我们却仍然坚持采用全面喷洒的方式。\\  

\noindent 33\\
The object of brush control along roads and rights-of-way is not to sweep the land clear of everything but grass; it is, rather, to eliminate plants ultimately tall enough to present an obstruction to drivers’ vision or interference with wires on rights-of-way. This means, in general, trees. Most shrubs are low enough to present no hazard; so, certainly, are ferns and wildflowers.
控制道路及公用设施用地周边灌木的目的,不是要把地面上除了草之外的所有植物都清除掉;而是要移除那些最终会长得足够高,从而遮挡司机视线,或干扰公用设施线路的植物。一般来说,这类植物指的是树木。大多数灌木足够低矮,不会造成危险;蕨类植物和野花当然也是如此。\\

\noindent 34\\
Selective spraying was developed by Dr. Frank Egler during a period of years at the American Museum of Natural History as director of a Committee for Brush Control Recommendations for Rights-of-Way. It took advantage of the inherent stability of nature, building on the fact that most communities of shrubs are strongly resistant to invasion by trees. By comparison, grasslands are easily invaded by tree seedlings. The object of selective spraying is not to produce grass on roadsides and rights-of-way but to eliminate the tall woody plants by direct treatment and to preserve all other vegetation. One treatment may be sufficient, with a possible follow-up for extremely resistant species; thereafter the shrubs assert control and the trees do not return. The best and cheapest controls for vegetation are not chemicals but other plants.\\
选择性喷洒法是弗兰克·埃格勒博士在美国自然历史博物馆任职多年期间,作为公用设施用地灌木控制建议委员会的主任研发出来的。该方法利用了大自然的内在稳定性,基于这样一个事实:大多数灌木群落对树木的入侵具有很强的抵抗力。相比之下,草原很容易被树苗侵占。选择性喷洒的目的不是要在路边和公用设施用地上种草,而是通过直接处理去除高大的木本植物,并保护所有其他植被。一次处理可能就足够了,对于极具抗性的物种,可能需要进行后续处理;此后,灌木会占据主导地位,树木也不会再生长回来。对植被最好且最经济的控制方法不是使用化学药剂,而是借助其他植物。 \\

\noindent 35\\
The method has been tested in research areas scattered throughout the eastern United States. Results show that once properly treated, an area becomes stabilized, requiring no re - spraying for at least 20 years. The spraying can often be done by men on foot, using knapsack sprayers, and having complete control over their material. Sometimes compressor pumps and material can be mounted on truck chassis, but there is no blanket spraying. Treatment is directed only to trees and any exceptionally tall shrubs that must be eliminated. The integrity of the environment is thereby preserved, the enormous value of the wildlife habitat remains intact, and the beauty of shrub and fern and wildflower has not been sacrificed.\\
这种方法已在美国东部各地的研究区域进行了测试。结果表明,一旦处理得当,一个区域就能稳定下来,至少20年都无需再次喷洒。通常,工作人员可以背着背包式喷雾器步行作业,完全掌控使用的药剂。有时,也可以将压缩机泵和药剂安装在卡车底盘上,但不会进行大面积的喷洒。处理对象仅针对树木以及必须清除的极高的灌木。这样一来,环境的完整性得以保留,野生动物栖息地的巨大价值也不受破坏,同时灌木、蕨类植物和野花的美丽也不会被牺牲。\\  

\noindent 36\\
Here and there the method of vegetation management by selective spraying has been adopted. For the most part, entrenched custom dies hard and blanket spraying continues to thrive, to exact its heavy annual costs from the taxpayer, and to inflict its damage on the ecological web of life. It thrives, surely, only because the facts are not known. When taxpayers understand that the bill for spraying the town roads should come due only once a generation instead of once a year, they will surely rise up and demand a change of method.\\
选择性喷洒这种植被管理方法已在一些地方得到应用。但在大多数情况下,根深蒂固的习惯很难改变,大面积喷洒的做法仍在盛行,每年都要让纳税人承担高昂的费用,还对生态生命网络造成破坏。这种做法之所以大行其道,肯定只是因为人们不了解实际情况。当纳税人明白,城镇道路的喷洒费用本应一代人支付一次,而不是每年都要支付时,他们肯定会站出来,要求改变这种方式。 \\

\noindent 37\\
Among the many advantages of selective spraying is the fact that it minimizes the amount of chemical applied to the landscape. There is no broadcasting of material but, rather, concentrated application to the base of the trees. The potential harm to wildlife is therefore kept to a minimum.\\
选择性喷洒有诸多优点,其中一点是它能将施用于自然景观的化学药剂用量减至最少。这种方式不是大面积撒施药剂,而是将药剂集中施用于树木基部。因此,对野生动物的潜在危害也被降到了最低。 \\

\noindent 38\\
The most widely used herbicides are 2,4-D, 2,4,5-T, and related compounds. Whether or not these are actually toxic is a matter of controversy. People spraying their lawns with 2,4-D and becoming wet with spray have occasionally developed severe neuritis and even paralysis. Although such incidents are apparently uncommon, medical authorities advise caution in use of such compounds. Other hazards, more obscure, may also attend the use of 2,4-D. It has been shown experimentally to disturb the basic physiological process of respiration in the cell, and to imitate X-rays in damaging the chromosomes. Some very recent work indicates that reproduction of birds may be adversely affected by these and certain other herbicides at levels far below those that cause death. \\
最广泛使用的除草剂是2,4-二氯苯氧乙酸(2,4-D)、2,4,5-三氯苯氧乙酸(2,4,5-T)以及相关化合物。这些除草剂是否真正有毒仍存在争议。使用2,4-D喷洒草坪时被药剂淋湿的人,偶尔会患上严重的神经炎,甚至瘫痪。尽管这类事件显然并不常见,但医学专家还是建议谨慎使用这类化合物。使用2,4-D还可能带来其他一些尚不明确的危害。实验表明,它会干扰细胞内呼吸的基本生理过程,并且像X射线一样对染色体造成损害。最近的一些研究表明,鸟类的繁殖可能会受到这些除草剂以及某些其他除草剂的不利影响,而这些影响发生时的药剂浓度远低于致死浓度。 \\

\noindent 39\\
Apart from any directly toxic effects, curious indirect results follow the use of certain herbicides. It has been found that animals, both wild herbivores and livestock, are sometimes strangely attracted to a plant that has been sprayed, even though it is not one of their natural foods. If a highly poisonous herbicide such as arsenic has been used, this intense desire to reach the wilting vegetation inevitably has disastrous results. Fatal results may follow, also, from less toxic herbicides if the plant itself happens to be poisonous or perhaps to possess thorns or burs. Poisonous range weeds, for example, have suddenly become attractive to livestock after spraying, and the animals have died from indulging this unnatural appetite. The literature of veterinary medicine abounds in similar examples: swine eating sprayed cockleburs with consequent severe illness, lambs eating sprayed thistles, bees poisoned by pasturing on mustard sprayed after it came into bloom. Wild cherry, the leaves of which are highly poisonous, has exerted a fatal attraction for cattle once its foliage has been sprayed with 2,4-D. Apparently the wilting that follows spraying (or cutting) makes the plant attractive. Ragwort has provided other examples. Livestock ordinarily avoid this plant unless forced to turn to it in late winter and early spring by lack of other forage. However, the animals eagerly feed on it after its foliage has been sprayed with 2,4-D. \\
除了直接的毒性作用外,使用某些除草剂还会产生一些奇怪的间接后果。人们发现,野生食草动物和家畜有时会奇怪地被喷洒过除草剂的植物所吸引,即使这些植物并非它们的天然食物。如果使用了像砷这样剧毒的除草剂,动物们对这些枯萎植物的强烈渴望必然会带来灾难性的后果。如果植物本身有毒,或者长有刺或芒刺,即使是毒性较低的除草剂也可能导致致命后果。例如,有毒的牧场杂草在喷洒除草剂后,突然对家畜产生了吸引力,而家畜因满足这种反常的食欲而死亡。兽医学文献中有大量类似的例子:猪吃了喷洒过除草剂的苍耳后患上重病;小羊吃了喷洒过除草剂的蓟草;蜜蜂在开花后被喷洒过除草剂的芥菜上采蜜而中毒。野樱桃的叶子有剧毒,其叶子被2,4 - D喷洒后,对牛产生了致命的吸引力。显然,喷洒(或砍伐)后植物的枯萎状态使其具有了吸引力。千里光也有类似的情况。家畜通常会避开这种植物,除非在冬末春初因缺乏其他饲料而不得不食用。然而,当千里光的叶子被2,4 - D喷洒后,家畜却会急切地食用。 \\

\noindent 40\\
The explanation of this peculiar behavior sometimes appears to lie in the changes which the chemical brings about in the metabolism of the plant itself. There is temporarily a marked increase in sugar content, making the plant more attractive to many animals.\\
对于这种奇特现象的一种解释是,化学药剂会改变植物自身的新陈代谢。喷洒药剂后,植物中的糖分含量会暂时显著增加,从而使它对许多动物更具吸引力。 \\

\noindent 41\\
Another curious effect of 2,4-D has important effects for livestock, wildlife, and apparently for men as well. Experiments carried out about a decade ago showed that after treatment with this chemical there is a sharp increase in the nitrate content of corn and of sugar beets. The same effect was suspected in sorghum, sunflower, spiderwort, lambs quarters, pigweed, and smartweed. Some of these are normally ignored by cattle, but are eaten with relish after treatment with 2,4-D. A number of deaths among cattle have been traced to sprayed weeds, according to some agricultural specialists. The danger lies in the increase in nitrates, for the peculiar physiology of the ruminant at once poses a critical problem. Most such animals have a digestive system of extraordinary complexity, including a stomach divided into four chambers. The digestion of cellulose is accomplished through the action of microorganisms (rumen bacteria) in one of the chambers. When the animal feeds on vegetation containing an abnormally high level of nitrates, the microorganisms in the rumen act on the nitrates to change them into highly toxic nitrites. Thereafter a fatal chain of events ensues: the nitrites act on the blood pigment to form a chocolate-brown substance in which the oxygen is so firmly held that it cannot take part in respiration, hence oxygen is not transferred from the lungs to the tissues. Death occurs within a few hours from anoxia, or lack of oxygen. The various reports of livestock losses after grazing on certain weeds treated with 2,4-D therefore have a logical explanation. The same danger exists for wild animals belonging to the group of ruminants, such as deer, antelope, sheep, and goats.\\
2,4-D的另一个奇特影响,对家畜、野生动物,显然对人类也有着重大影响。大约十年前开展的实验表明使用这种化学药剂处理后,玉米和甜菜中的硝酸盐含量会急剧增加。人们怀疑高粱、向日葵、紫露草、藜、 猪毛菜和蓼属植物也会出现同样的情况。其中一些植物通常牛是不会吃的,但在经过2,4-D处理后,牛却吃得津津有味。据一些农业专家称,已经发现多起牛的死亡事件与食用喷洒过药剂的杂草有关。危险源于硝酸盐含量的增加,因为反刍动物特殊的生理机能会立刻引发一个关键问题。大多数反刍动物拥有极其复杂的消化系统,包括一个分为四个腔室的胃。纤维素的消化是通过其中一个腔室中的微生物(瘤胃细菌)的作用来完成的。当动物食用含有异常高含量硝酸盐的植物时,瘤胃中的微生物会作用于硝酸盐,将它们转化为剧毒的亚硝酸盐。随后,一系列致命事件接踵而至:亚硝酸盐作用于血液中的色素,形成一种巧克力棕色的物质,其中的氧被紧紧束缚,无法参与呼吸作用,因此氧气无法从肺部输送到组织。动物会在几小时内因缺氧而死亡。因此,各种关于家畜在食用经2,4-D处理过的特定杂草后死亡的报告,都有了合理的解释。对于鹿、羚羊、绵羊和山羊等反刍类野生动物来说,同样存在这种危险。 \\

\noindent 42\\
Although various factors (such as exceptionally dry weather) can cause an increase in nitrate content, the effect of the soaring sales and applications of 2,4-D cannot be ignored. The situation was considered important enough by the University of Wisconsin Agricultural Experiment Station to justify a warning in 1957 that “plants killed by 2,4-D may contain large amounts of nitrate.” The hazard extends to human beings as well as animals and may help to explain the recent mysterious increase in “silo deaths.” When corn, oats, or sorghum containing large amounts of nitrates are ensiled they release poisonous nitrogen oxide gases, creating a deadly hazard to anyone entering the silo. Only a few breaths of one of these gases can cause a diffuse chemical pneumonia. In a series of such cases studied by the University of Minnesota Medical School all but one terminated fatally. \\
虽然各种因素(比如异常干燥的天气)会导致硝酸盐含量增加,但2,4-D的销量猛增和广泛使用所带来的影响也不容忽视。威斯康星大学农业实验站认为这一情况极为重要,因此在1957年发出警告:“被2,4-D杀死的植物可能含有大量硝酸盐”。这种危害不仅影响动物,也波及人类,或许还能解释近年来“谷仓死亡事件”莫名增多的现象。当含有大量硝酸盐的玉米、燕麦或高粱被青贮时,会释放出有毒的氧化氮气体,对进入谷仓的人构成致命威胁。只需吸入少量这类气体,就能引发弥漫性化学性肺炎。明尼苏达大学医学院研究的一系列此类案例中,几乎无一例外以死亡告终。\\

\noindent 43\\
“Once again we are walking in nature like an elephant in the china cabinet.” So C. J. Briejer, a Dutch scientist of rare understanding, sums up our use of weed killers. “In my opinion too much is taken for granted. We do not know whether all weeds in crops are harmful or whether some of them are useful,” says Dr. Briejer.\\
“我们又一次像在瓷器店里的大象一样,在大自然中肆意妄为。” 荷兰科学家C. J. 布里杰见解独到,他如此总结我们对除草剂的使用。布里杰博士说:“在我看来,我们把太多事情都视作理所当然了。我们并不清楚作物中的所有杂草是否都有害,或者其中一些是否有用。” \\

\noindent 44\\
Seldom is the question asked, What is the relation between the weed and the soil? Perhaps, even from our narrow standpoint of direct self-interest, the relation is a useful one. As we have seen, soil and the living things in and upon it exist in a relation of interdependence and mutual benefit. Presumably the weed is taking something from the soil; perhaps it is also contributing something to it. A practical example was provided recently by the parks in a city in Holland. The roses were doing badly. Soil samples showed heavy infestations by tiny nematode worms. Scientists of the Dutch Plant Protection Service did not recommend chemical sprays or soil treatments; instead, they suggested that marigolds be planted among the roses. This plant, which the purist would doubtless consider a weed in any rosebed, releases an excretion from its roots that kills the soil nematodes. The advice was taken; some beds were planted with marigolds, some left without as controls. The results were striking. With the aid of the marigolds the roses flourished; in the control beds they were sickly and drooping. Marigolds are now used in many places for combating nematodes.\\
很少有人会问,杂草与土壤之间有什么关系? 或许,即便从我们只顾自身直接利益的狭隘角度来看,这种关系也是有益的。 正如我们所见,土壤以及其中和表面的生物,处于一种相互依存、互惠互利的关系中。 想必杂草从土壤中获取了某些东西,或许它也在回馈土壤。 荷兰一座城市的公园最近就给出了一个实例。那里的玫瑰生长状况不佳,土壤样本显示受到了大量微小线虫的侵扰。荷兰植物保护局的科学家并不建议使用化学喷剂或进行土壤处理,相反,他们建议在玫瑰丛中种植万寿菊。纯粹主义者无疑会认为,在任何玫瑰园里,万寿菊都是一种杂草,但它的根部会释放出一种分泌物,能够杀死土壤中的线虫。这个建议被采纳了,一些花坛种上了万寿菊,一些则作为对照不种。结果十分显著。在万寿菊的帮助下,玫瑰茁壮成长;而在对照花坛中,玫瑰却病恹恹且无精打采 。如今,万寿菊在许多地方被用于防治线虫。 \\

\noindent 45\\
In the same way, and perhaps quite unknown to us, other plants that we ruthlessly eradicate may be performing a function that is necessary to the health of the soil. One very useful function of natural plant communities—now pretty generally stigmatized as “weeds”—is to serve as an indicator of the condition of the soil. This useful function is of course lost where chemical weed killers have been used.\\
同样地,或许我们全然不知,那些被我们无情铲除的其他植物,可能对土壤的健康起着至关重要的作用。自然植物群落(如今大多被蔑称为 “杂草”)有一个非常实用的功能,那就是作为土壤状况的指示物。当然,在使用化学除草剂的地方,这个实用功能就丧失了。 \\

\noindent 46\\
Those who find an answer to all problems in spraying also overlook a matter of great scientific importance—the need to preserve some natural plant communities. We need these as a standard against which we can measure the changes our own activities bring about. We need them as wild habitats in which original populations of insects and other organisms can be maintained, for, as will be explained in Chapter 16, the development of resistance to insecticides is changing the genetic factors of insects and perhaps other organisms. One scientist has even suggested that some sort of “zoo” should be established to preserve insects, mites, and the like, before their genetic composition is further changed.\\
那些认为喷洒(除草剂等)能解决所有问题的人,也忽视了一个具有重大科学意义的问题,即保护某些自然植物群落的必要性。我们需要这些自然植物群落作为一个标准,用以衡量我们自身活动所带来的变化。我们还需要它们作为野生栖息地,以维持昆虫和其他生物的原始种群数量,因为,正如第16章将会阐述的那样,昆虫对杀虫剂产生抗药性这一现象正在改变昆虫以及可能其他生物的遗传因素。一位科学家甚至建议,应该建立某种 “动物园”,以便在昆虫、螨虫等生物的遗传构成进一步改变之前对它们加以保护。 \\

\noindent 47\\
Some experts warn of subtle but far-reaching vegetational shifts as a result of the growing use of herbicides. The chemical 2,4-D, by killing out the broad-leaved plants, allows the grasses to thrive in the reduced competition—now some of the grasses themselves have become “weeds,” presenting a new problem in control and giving the cycle another turn. This strange situation is acknowledged in a recent issue of a journal devoted to crop problems: “With the widespread use of 2,4-D to control broadleaved weeds, grass weeds in particular have increasingly become a threat to corn and soybean yields.”\\
一些专家警告称,随着除草剂使用的日益增多,植被会发生微妙但影响深远的变化。化学药剂2,4-D通过杀死阔叶植物,使得草本植物在竞争减少的环境中得以茁壮成长,如今,一些草本植物本身也成了 “杂草”,这带来了新的控制难题,让(杂草控制的)循环再次转动。一份专门研究作物问题的期刊最近一期中提到了这种奇怪的情况:“随着2,4-D被广泛用于控制阔叶杂草,草类杂草尤其日益成为影响玉米和大豆产量的威胁。” \\

\noindent 48\\
Ragweed, the bane of hay fever sufferers, offers an interesting example of the way efforts to control nature sometimes boomerang. Many thousands of gallons of chemicals have been discharged along roadsides in the name of ragweed control. But the unfortunate truth is that blanket spraying is resulting in more ragweed, not less. Ragweed is an annual; its seedlings require open soil to become established each year. Our best protection against this plant is therefore the maintenance of dense shrubs, ferns, and other perennial vegetation. Spraying frequently destroys this protective vegetation and creates open, barren areas which the ragweed hastens to fill. It is probable, moreover, that the pollen content of the atmosphere is not related to roadside ragweed, but to the ragweed of city lots and fallow fields.\\
豚草是花粉热患者的克星,它是一个有趣的例子,说明人类控制自然的努力有时会适得其反。为了控制豚草,成千上万加仑的化学药剂被喷洒在路边。但不幸的是,大面积喷洒药剂的结果是豚草数量增多,而非减少。豚草是一年生植物,其幼苗每年都需要开阔的土壤才能生长。因此,对付这种植物的最佳办法是保持茂密的灌木、蕨类植物和其他多年生植被。喷洒药剂常常会破坏这些起保护作用的植被,形成开阔、荒芜的区域,而豚草会迅速占据这些地方。此外,大气中的花粉含量很可能与路边的豚草无关,而是与城市空地和休耕地上的豚草有关。 \\

\noindent 49\\
The booming sales of chemical crabgrass killers are another example of how readily unsound methods catch on. There is a cheaper and better way to remove crabgrass than to attempt year after year to kill it out with chemicals. This is to give it competition of a kind it cannot survive, the competition of other grass. Crabgrass exists only in an unhealthy lawn. It is a symptom, not a disease in itself. By providing a fertile soil and giving the desired grasses a good start, it is possible to create an environment in which crabgrass cannot grow, for it requires open space in which it can start from seed year after year.\\
化学防除狗牙根( crabgrass ,又称牛筋草)药剂的热销,是不合理方法轻易流行起来的又一个例子。有一个比年复一年试图用化学药剂来除掉狗牙根更便宜、更好的方法,那就是为它制造一种使其无法生存的竞争环境,即来自其他草类的竞争。狗牙根只生长在不健康的草坪中,它只是一种表象,而不是一种独立的 “病害”。通过提供肥沃的土壤,并帮助理想的草种良好生长,就有可能营造出一个狗牙根无法生长的环境,因为狗牙根需要开阔的空间,以便每年从种子开始生长。 \\

\noindent 50\\
Instead of treating the basic condition, suburbanites—advised by nurserymen who in turn have been advised by the chemical manufacturers—continue to apply truly astonishing amounts of crabgrass killers to their lawns each year. Marketed under trade names which give no hint of their nature, many of these preparations contain such poisons as mercury, arsenic, and chlordane. Application at the recommended rates leaves tremendous amounts of these chemicals on the lawn. Users of one product, for example, apply 60 pounds of technical chlordane to the acre if they follow directions. If they use another of the many available products, they are applying 175 pounds of metallic arsenic to the acre. The toll of dead birds, as we shall see in Chapter 8, is distressing. How lethal these lawns may be for human beings is unknown.\\
郊区居民在苗木商的建议下(而苗木商又是听从了化学药剂制造商的建议),并没有去改善根本状况,而是每年继续在草坪上使用数量惊人的狗牙根除草剂。这些除草剂以一些无法体现其性质的商品名进行销售,其中许多制剂含有汞、砷和氯丹等有毒物质。按照推荐用量使用后,草坪上会残留大量这些化学物质。例如,一种产品的使用者如果按照说明操作,每英亩会使用60磅工业级氯丹。如果使用市面上的另一种产品,每英亩则会施用175磅金属砷。正如我们将在第8章看到的,鸟类死亡的数量令人痛心。而这些草坪对人类的危害程度究竟有多大,目前还不得而知。 \\

\noindent 51\\
The success of selective spraying for roadside and right-of-way vegetation, where it has been practiced, offers hope that equally sound ecological methods may be developed for other vegetation programs for farms, forests, and ranges—methods aimed not at destroying a particular species but at managing vegetation as a living community.\\
在一些地方,针对路边和道路用地植被的选择性喷洒已取得成功,这带来了一种希望,即有可能为农场、森林和牧场的其他植被管理项目开发出同样合理的生态方法——这些方法的目的不是消灭某个特定物种,而是将植被作为一个有生命的群落来进行管理。 \\

\noindent 52\\
Other solid achievements show what can be done. Biological control has achieved some of its most spectacular successes in the area of curbing unwanted vegetation. Nature herself has met many of the problems that now beset us, and she has usually solved them in her own successful way. Where man has been intelligent enough to observe and to emulate Nature he, too, is often rewarded with success.\\
其他切实的成果表明了我们能够做到什么。在控制有害植被方面,生物防治已经取得了一些极为显著的成功。大自然本身已经应对过许多如今困扰我们的问题,而且通常以她自己成功的方式解决了这些问题。只要人类足够明智,去观察并模仿大自然,往往也能收获成功。 \\

\noindent 53\\
An outstanding example in the field of controlling unwanted plants is the handling of the Klamath-weed problem in California. Although the Klamath weed, or goatweed, is a native of Europe (where it is called St. Johnswort), it accompanied man in his westward migrations, first appearing in the United States in 1793 near Lancaster, Pennsylvania. By 1900 it had reached California in the vicinity of the Klamath River, hence the name locally given to it. By 1929 it had occupied about 100,000 acres of rangeland, and by 1952 it had invaded some two and one half million acres.\\
在控制有害植物领域,一个突出的例子是加利福尼亚州对克拉马斯草(Klamath-weed )问题的处理。尽管克拉马斯草, 又称山羊草,原产于欧洲(在欧洲被称为贯叶连翘),但它随着人类向西迁徙,于1793年首次出现在美国宾夕法尼亚州兰卡斯特附近。到1900年,它已蔓延至加利福尼亚州克拉马斯河附近,这就是它当地名称的由来。到1929年,它占据了约10万英亩的牧场,而到1952年,它的入侵面积已达到约250万英亩。 \\

\noindent 54\\
Klamath weed, quite unlike such native plants as sagebrush, has no place in the ecology of the region, and no animals or other plants require its presence. On the contrary, wherever it appeared livestock became “scabby, sore-mouthed, and unthrifty” from feeding on this toxic plant. Land values declined accordingly, for the Klamath weed was considered to hold the first mortgage.\\
克拉马斯草与山艾树等本土植物截然不同,它在当地的生态系统中毫无立足之地,也没有动物或其他植物依赖于它而生存。相反,无论这种有毒植物出现在哪里,家畜一旦食用就会变得 “皮肤长痂、口部生疮且发育不良” 。土地价值也随之下降,因为克拉马斯草被视为(导致土地贬值的)首要因素。 \\

\noindent 55\\
In Europe the Klamath weed, or St. Johnswort, has never become a problem because along with the plant there have developed various species of insects; these feed on it so extensively that its abundance is severely limited. In particular, two species of beetles in southern France, pea-sized and of metallic color, have their whole beings so adapted to the presence of the weed that they feed and reproduce only upon it.\\
在欧洲,克拉马斯草, 即贯叶连翘,从来没有成为过问题,因为伴随着这种植物出现了各种各样的昆虫,这些昆虫大量以它为食,使其数量受到了严格的限制。尤其是在法国南部的两种甲虫, 豌豆大小,颜色如金属般闪亮,它们的整个生存习性都高度适应了这种杂草的存在,以至于只以它为食并在其上繁衍后代。 \\

\noindent 56\\
It was an event of historic importance when the first shipments of these beetles were brought to the United States in 1944, for this was the first attempt in North America to control a plant with a plant - eating insect. By 1948 both species had become so well established that no further importations were needed. Their spread was accomplished by collecting beetles from the original colonies and redistributing them at the rate of millions a year. Within small areas the beetles accomplish their own dispersion, moving on as soon as the Klamath weed dies out and locating new stands with great precision. And as the beetles thin out the weed, desirable range plants that have been crowded out are able to return.\\
1944 年,首批这类甲虫被运往美国,这是一个具有历史意义的事件,因为这是北美首次尝试利用食草昆虫来控制植物。到 1948 年,这两种甲虫都已很好地繁衍开来,无需再进口。它们的扩散是通过从最初引入的种群中收集甲虫,然后以每年数百万只的速度重新分发来实现的。在小范围内,甲虫会自行扩散,一旦克拉马斯草消失,它们就会迁徙,并能极其精准地找到新的草群。随着甲虫使克拉马斯草的数量减少,原本被排挤的优质牧场植物得以重新生长。 \\

\noindent 57\\
A ten-year survey completed in 1959 showed that control of the Klamath weed had been “more effective than hoped for even by enthusiasts,” with the weed reduced to a mere 1 per cent of its former abundance. This token infestation is harmless and is actually needed in order to maintain a population of beetles as protection against a future increase in the weed.\\
1959年完成的一项为期十年的调查显示,对克拉马斯草的控制效果 “甚至比乐观者所期望的还要好”,这种草的数量已减少到原来的1\%。这点少量的侵扰并无危害,实际上,为了维持甲虫的种群数量,以防止日后克拉马斯草数量再次增加,这点草是必要的。 \\

\noindent 58\\
Another extraordinarily successful and economical example of weed control may be found in Australia. With the colonists’ usual taste for carrying plants or animals into a new country, a Captain Arthur Phillip had brought various species of cactus into Australia about 1787, intending to use them in culturing cochineal insects for dye. Some of the cacti or prickly pears escaped from his gardens and by 1925 about 20 species could be found growing wild. Having no natural controls in this new territory, they spread prodigiously, eventually occupying about 60 million acres. At least half of this land was so densely covered as to be useless.\\
在澳大利亚可以找到另一个极为成功且经济的杂草控制范例。由于殖民者向来喜欢把植物或动物带到新的国度,1787 年左右,亚瑟·菲利普船长将各种仙人掌引入澳大利亚,打算用它们来养殖胭脂虫以制作染料。一些仙人掌从他的花园中逸出,到 1925 年,大约有 20 个品种的仙人掌在野外生长。在这个新的环境中没有自然的制约因素,它们大量繁殖,最终占据了约 6000 万英亩的土地。这片土地中至少有一半被仙人掌覆盖得过于茂密,以至于无法利用。 \\

\noindent 59\\
In 1920 Australian entomologists were sent to North and South America to study insect enemies of the prickly pears in their native habitat. After trials of several species, 3 billion eggs of an Argentine moth were released in Australia in 1930. Seven years later the last dense growth of the prickly pear had been destroyed and the once uninhabitable areas reopened to settlement and grazing. The whole operation had cost less than a penny per acre. In contrast, the unsatisfactory attempts at chemical control in earlier years had cost about £10 per acre.\\
1920年,澳大利亚的昆虫学家被派往北美洲和南美洲,去研究仙人掌在其原生栖息地的昆虫天敌。在对几个物种进行试验后,1930年,30亿枚阿根廷蛾的卵被投放到澳大利亚。七年后,最后一片茂密生长的仙人掌被清除,曾经无法居住的地区重新可供人们定居和放牧。整个行动每英亩的成本不到一便士。相比之下,早些年采用化学方法控制仙人掌但效果不佳,每英亩的成本约为10英镑。 \\

\noindent 60\\
Both of these examples suggest that extremely effective control of many kinds of unwanted vegetation might be achieved by paying more attention to the role of plant - eating insects. The science of range management has largely ignored this possibility, although these insects are perhaps the most selective of all grazers and their highly restricted diets could easily be turned to man’s advantage.\\
这两个例子都表明,更多地关注食草昆虫的作用,或许能极其有效地控制多种有害植被。牧场管理科学在很大程度上忽视了这种可能性,尽管这些昆虫可能是所有食草动物中最具选择性的,而且它们极为特定的食物偏好很容易为人类所用。\\


\end{document}