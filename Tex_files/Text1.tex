\documentclass{article}
\usepackage[utf8]{inputenc} % 支持UTF-8编码
\usepackage{xeCJK} % 支持中文
\usepackage{graphicx} % 引入用于插入图片的宏包
\usepackage{hyperref} % 引入超链接宏包
\usepackage{amsmath}

\usepackage{geometry}
\geometry{
  a4paper,
  left=20mm,
  right=20mm,
  top=30mm,
  bottom=30mm
}

% 设置行距为1.5倍
\usepackage{setspace}
\linespread{1.75}

\begin{document}

\title{\textbf{
GFN1000 与自然对话\\
中英双语对照翻译版
}} % 文章标题
\date{}
\maketitle % 生成标题

\setcounter{secnumdepth}{0} % 禁止章节编号,但仍添加到目录
\tableofcontents
\newpage

\section{Text 1a from Republic / \textit{Plato}}
SOCRATES’ NARRATION CONTINUES:\\
1\\
\textbf{SOCRATES:} Next, then, compare the effect of education and that of the lack of it on our nature to an experience like this. Imagine human beings living in an underground, cavelike dwelling, with an entrance a long way up that is open to the light and as wide as the cave itself. They have been there since childhood, with their necks and legs fettered, so that they are fixed in the same place, able to see only in front of them, because their fetter prevents them from turning their heads around. Light is provided by a fire burning far above and behind them. Between the prisoners and the fire, there is an elevated road stretching. Imagine that along this road a low wall has been built—like the screen in front of people that is provided by puppeteers, and above which they show their puppets.\\
\textbf{GLAUCON:} I am imagining it.\\
\textbf{苏格拉底:}接下来,将教育的效果以及缺乏教育对我们本性的影响与这样的经历相比较。想象一下,人类生活在地下洞穴般的住所中,有一个远离地面的入口,这个入口向光线敞开,并且和洞穴本身一样宽。他们从小就在那里,脖子和腿被束缚,以至于他们被固定在同一地点,只能看到他们面前的东西,因为他们的束缚阻止了他们转头。光线是由远在他们上方和后方燃烧的火提供的。在囚犯和火之间,有一条高路延伸。想象一下,在这条路上建了一堵低墙——就像木偶师在人们面前提供屏幕,他们在上面展示他们的木偶。\\
\textbf{格劳孔:}我正在想象。\\
\\\noindent2\\
\textbf{SOCRATES:} Also imagine, then, that there are people alongside the wall carrying multifarious artifacts that project above it—statues of people and other animals, made of stone, wood, and every material. And as you would expect, some of the carriers are talking and some are silent.\\
GLAUCON: It is a strange image you are describing, and strange prisoners.\\
\textbf{苏格拉底:}再想象一下,墙边还有人携带着各种各样的物品,这些物品投射在墙上——有人和其他动物的雕像,它们是由石头、木头和各种材料制成的。正如你所料,有些携带者在说话,有些则保持沉默。\\
格劳孔:你描述的是一个奇怪的景象,囚犯人也很奇怪。\\
\\\noindent3\\
\textbf{SOCRATES:} They are like us. I mean, in the first place, do you think these prisoners have ever seen anything of themselves and one another besides the shadows that the fire casts on the wall of the cave in front of them?
\textbf{GLAUCON: }How could they, if they have to keep their heads motionless throughout life?\\
\textbf{苏格拉底:}他们就像我们一样。我的意思是,首先,你认为这些囚犯人除了火光在他们面前的洞穴墙壁上投射的影子之外,还见过他们自己或其他人吗?\\
\textbf{格劳孔:}如果他们一生都要保持头部不动,他们怎么可能见过呢?\\
\\
\noindent4\\
\textbf{SOCRATES: }What about the things carried along the wall? Isn't the same true where they are concerned?\\
\textbf{GLAUCON:} Of course.\\
\textbf{苏格拉底:}那么,那些沿着墙壁携带的物品呢?对于它们来说,情况不也是一样的吗?
\textbf{格劳孔:}当然。\\
\\
\noindent5\\
\textbf{SOCRATES: }And if they could engage in discussion with one another, don't you think they would assume that the words they used applied to the things they see passing in front of them?\\
\textbf{GLAUCON: }They would have to.\\
\textbf{苏格拉底:}如果他们能够相互讨论,难道你不认为他们会认为他们所使用的词汇适用于他们看到在面前经过的事物吗?\\
\textbf{格劳孔:}他们不得不这样认为。\\
\\\noindent6\\
\textbf{SOCRATES: }What if their prison also had an echo from the wall facing them? When one of the carriers passing along the wall spoke, do you think they would believe that anything other than the shadow passing in front of them was speaking?\\
\textbf{GLAUCON: }I do not, by Zeus.\\
\textbf{苏格拉底:}如果他们的监狱也有来自面对他们的墙壁的回声呢?当沿着墙壁走过的搬运者之一说话时,你认为他们会相信除了在他们面前经过的影子之外还有其他东西在说话吗?\\
\textbf{格劳孔:}我不会这么认为,我发誓。\\
\\\noindent7\\
\textbf{SOCRATES: }All in all, then, what the prisoners would take for true reality is nothing other than the shadows of those artifacts.\\
\textbf{GLAUCON: }That's entirely inevitable.\\
\textbf{苏格拉底:}总而言之,囚犯们所认为的真实现实不过是那些物品的影子。\\
\textbf{格劳孔:}那是完全不可避免的。\\
\\\noindent8\\
\textbf{SOCRATES: }Consider, then, what being released from their bonds and cured of their foolishness would naturally be like, if something like this should happen to them. When one was freed and suddenly compelled to stand up, turn his neck around, walk, and look up toward the light, he would be pained by doing all these things and be unable to see the things whose shadows he had seen before, because of the flashing lights. What do you think he would say if we told him that what he had seen before was silly nonsense, but that now—because he is a bit closer to what is, and is turned toward things that are more—he sees more correctly? And in particular, if we pointed to each of the things passing by and compelled him to answer what each of them is, don’t you think he would be puzzled and believe that the things he saw earlier were more truly real than the ones he was being shown?\\
\textbf{GLAUCON:} Much more so.\\
\textbf{苏格拉底:}想象一下,如果他们中的某人被解除了束缚,摆脱了愚昧,自然会发生什么。当一个人被解放,并突然被迫站起来,转过头,走动,抬头看向光源,他会因为做这些事情而感到痛苦,并且无法看到他以前见过的影子所代表的事物,因为强光闪烁。如果我们告诉他,他以前所见的一切都是愚蠢的无稽之谈,但现在——因为他更接近真实,并且转向了更真实的东西——他看得更清楚了,你认为他会怎么说?特别是,如果我们指着每一件经过的事物并强迫他回答每件事物是什么,你不觉得他会困惑,并相信他以前看到的东西比现在展示给他的更真实吗?\\
\textbf{格劳孔:}确实如此。\\
\\\noindent9\\
\textbf{SOCRATES: }And if he were compelled to look at the light itself, wouldn't his eyes be pained and wouldn't he turn around and flee toward the things he is able to see, and believe that they are really clearer than the ones he is being shown? \\
\textbf{GLAUCON}\textbf{: }He would.\\
\textbf{苏格拉底:}如果他被迫看向光源本身,他的眼睛不会感到疼痛,并且他不会转过身去奔向他能看到的事物,并相信这些事物真的比他被展示的更清晰吗? \\
\textbf{格劳孔:}他会的。\\
\\\noindent10\\
\textbf{SOCRATES:} And if someone dragged him by force away from there, along the rough, steep, upward path, and did not let him go until he had dragged him into the light of the sun, wouldn't he be pained and angry at being treated that way? And when he came into the light, wouldn't he have his eyes filled with sunlight and be unable to see a single one of the things now said to be truly real?\\
\textbf{GLAUCON: }No, he would not be able to—at least not right away.\\
\textbf{苏格拉底:}如果有人强行将他从那里拖走,沿着崎岖、陡峭、向上的道路,并且不让他离开,直到将他拖到阳光下,他难道不会因为受到这样的对待而感到痛苦和愤怒吗?当他来到阳光下时,他的眼睛不会被阳光充满,而无法看到任何一个现在被认为是真正真实的东西吗?\\
\textbf{格劳孔:}不,他不会能够——至少不会立刻能够。\\

\noindent11\\
\textbf{SOCRATES:}He would need time to get adjusted, I suppose, if he is going to see the things in the world above. At first, he would see shadows most easily, then images of men and other things in water, then the things themselves. From these, it would be easier for him to go on to look at the things in the sky and the sky itself at night, gazing at the light of the stars and the moon, than during the day, gazing at the sun and the light of the sun.\\
GLAUCON: Of course.\\
\textbf{苏格拉底:}我想,如果他要看见上面世界的事物,他需要时间来适应。起初,他最容易看见影子,然后是水中人和其他事物的影像,接着是事物本身。从这些开始,对他来说,在夜晚凝视星星和月亮的光芒,进而观看天空中的事物和天空本身,会比在白天凝视太阳及其光芒更容易。\\
\textbf{格劳孔:}当然。\\

\noindent12\\
\textbf{SOCRATES:}Finally, I suppose, he would be able to see the sun—not reflections of it in water or some alien place, but the sun just by itself in its own place—and be able to look at it and see what it is like.\\
\textbf{GLAUCON: }Necessarily.\\
\textbf{苏格拉底:}最后,我想,他将能够看到太阳 —— 不是水中或某个陌生地方的太阳倒影,而是太阳本身在它自己的位置上 —— 并且能够注视它,看清它是什么样子。\\
\textbf{格劳孔:}必然如此。\\

\noindent13\\
\textbf{SOCRATES:}After that, he would already be able to conclude about it that it provides the seasons and the years, governs everything in the visible world, and is in some way the cause of all the things that he and his fellows used to see.\\
\textbf{GLAUCON: }That would clearly be his next step.\\
\textbf{苏格拉底:}在那之后,他将能够推断出,太阳带来了四季和年岁,主宰着可见世界中的一切,并且在某种程度上是他和他的同伴过去所见到的所有事物的成因。\\
\textbf{格劳孔:}那显然会是他接下来的想法。\\

\noindent14\\
SOCRATES: What about when he reminds himself of his first dwelling - place, what passed for wisdom there, and his fellow prisoners? Don't you think he would count himself happy for the change and pity the others?\\
GLAUCON: Certainly.\\
\textbf{苏格拉底:}当他回想起自己最初的栖身之所,那里被当作智慧的东西,以及他的那些狱友时,你难道不认为他会为这种改变而觉得自己幸福,并怜悯其他人吗?\\
\textbf{格劳孔:}当然会。\\

\noindent15\\
\textbf{SOCRATES:}And if there had been honors, praises, or prizes among them for the one who was sharpest at identifying the shadows as they passed by; and was best able to remember which usually came earlier, which later, and which simultaneously; who was thus best able to prophesize the future, do you think that our man would desire these rewards or envy those among the prisoners who were honored and held power? Or do you think he would feel with Homer that he would much prefer to “work the earth as a serf for another man, a man without possessions of his own,” and go through any sufferings, rather than share their beliefs and live as they do?\\
\textbf{GLAUCON: }Yes, I think he would rather suffer anything than live like that.\\
\textbf{苏格拉底:}而且,如果在他们当中,对于那个在影子掠过时最敏锐地辨认出影子,最善于记住哪些影子通常先出现、哪些后出现、哪些同时出现,因而最能预言未来的人,设有荣誉、赞扬或奖品,你认为我们所说的这个人会渴望这些奖赏,或者嫉妒那些在囚犯中受到尊敬且握有权力的人吗?或者,你认为他会像荷马所说的那样,宁愿 “给一个没有财产的人当雇工去耕地”,承受任何苦难,也不愿认同他们的观念并像他们那样生活吗?\\
\textbf{格劳孔:}是的,我认为他宁愿承受任何苦难也不愿那样生活。\\


\noindent16\\
\textbf{SOCRATES:}Consider this too, then. If this man went back down into the cave and sat down in his same seat, wouldn't his eyes be filled with darkness, coming suddenly out of the sun like that?\\
\textbf{GLAUCON: }Certainly.\\
\textbf{苏格拉底:}那么,再想想这个。如果这个人回到洞穴,又坐在他原来的位置上,就那样突然从阳光下回来,他的眼睛难道不会充满黑暗吗?\\
\textbf{格劳孔:}肯定会。\\

\noindent17\\
\textbf{SOCRATES:}Now, if he had to compete once again with the perpetual prisoners in recognizing the shadows, while his sight was still dim and before his eyes had recovered, and if the time required for readjustment was not short, wouldn't he provoke ridicule? Wouldn't it be said of him that he had returned from his upward journey with his eyes ruined, and that it is not worthwhile even to try to travel upward? And as for anyone who tried to free the prisoners and lead them upward, if they could somehow get their hands on him, wouldn't they kill him?\\
\textbf{GLAUCON: }They certainly would.\\
\textbf{苏格拉底:}现在,如果他的视力仍然模糊,在眼睛恢复之前,又不得不再次与那些一直被困在洞穴里的人竞争辨认影子,而且重新适应所需的时间又不短,他难道不会遭到嘲笑吗?难道不会有人说他从上面的旅程回来,眼睛坏掉了,甚至尝试向上走都是不值得的吗?而且,要是有谁试图解救这些囚犯并把他们带上去,如果他们能设法抓住这个人,难道不会杀了他吗?\\
\textbf{格劳孔:}他们肯定会的。\\

\noindent18\\
\textbf{SOCRATES:}This image, my dear Glaucon, must be fitted together as a whole with what we said before. The realm revealed through sight should be likened to the prison dwelling, and the light of the fire inside it to the sun's power. And if you think of the upward journey and the seeing of things above as the upward journey of the soul to the intelligible realm, you won't mistake my intention—since it is what you wanted to hear about. Only the god knows whether it is true. But this is how these phenomena seem to me: in the knowable realm, the last thing to be seen is the form of the good, and it is seen only with toil and trouble. Once one has seen it, however, one must infer that it is the cause of all that is correct and beautiful in anything, that in the visible realm it produces both light and its source, and that in the intelligible realm it controls and provides truth and understanding; and that anyone who is to act sensibly in private and public must see it.\\
\textbf{GLAUCON: }I agree, so far as I am able.\\
\textbf{苏格拉底:}我亲爱的格劳孔,这个比喻必须作为一个整体与我们之前所说的内容联系起来。通过视觉展现的领域应被比作牢狱居所,其中火光的力量应被比作太阳的力量。如果你把向上的旅程以及对上面事物的观照,看作是灵魂向可知领域的上升之旅,你就不会误解我的意图 —— 因为这正是你想听的内容。只有神知道这是否真实。但在我看来这些现象是这样的:在可知领域中,最后要看到的是善的理念,而且只有历经艰辛才能看到它。然而,一旦有人看到了它,就必须推断出它是一切事物中正确与美好之物的原因,在可见领域中它产生了光及其源头,在可知领域中它掌控并提供了真理与理解;并且任何想要在私人和公共事务中明智行事的人都必须看到它。\\
\textbf{格劳孔:}就我所能理解的范围,我同意。\\

\noindent19\\
\textbf{SOCRATES:}Come on, then, and join me in this further thought: you should not be surprised that the ones who get to this point are not willing to occupy themselves with human affairs, but that, on the contrary, their souls are always eager to spend their time above. I mean, that is surely what we would expect, if indeed the image described before is also accurate here.\\
\textbf{GLAUCON: }It is what we would expect.\\
\textbf{苏格拉底:}那么,来吧,和我一起进一步思考:你不应该对那些达到这种境界的人不愿意投身于人间事务感到惊讶,相反,他们的灵魂总是渴望在上面度过时光。我的意思是,如果之前所描述的比喻在这里同样准确的话,那肯定是我们所预料的情况。\\
\textbf{格劳孔:}这正是我们所预料的。\\

\noindent20\\
\textbf{SOCRATES:}What about when someone, coming from looking at divine things, looks to the evils of human life? Do you think it is surprising that he behaves awkwardly and appears completely ridiculous, if—while his sight is still dim and he has not yet become accustomed to the darkness around him—he is compelled, either in the courts or elsewhere, to compete about the shadows of justice, or about the statues of which they are the shadows; and to dispute the way these things are understood by people who have never seen justice itself?\\
\textbf{GLAUCON: }It is not surprising at all.\\
\textbf{苏格拉底:}当一个人从凝视神圣事物转而注视人类生活的丑恶时会怎样呢?你难道不觉得,如果在他视力仍然模糊,还未习惯周围的黑暗时,就被迫在法庭或其他地方,去争论正义的影子,或者正义影子所对应的雕像,并与那些从未见过正义本身的人争论对这些事物的理解方式,他举止笨拙,显得十分可笑,这并不奇怪吗?\\
\textbf{格劳孔:}一点也不奇怪。\\

\newpage
\section{Text 1b from The Beginnings of Western Science/ \textit{David C. Lindberg}}
\textbf{Plato’s World of Forms}\\
\\\noindent28\\
The death of Socrates in 399 B.C., coming as it did around the turn of the century (not on their calendar, of course, but on ours), has made it a convenient point of demarcation in the history of Greek philosophy. Thus Socrates’ predecessors of the sixth and fifth centuries (the philosophers who have occupied us until now in this chapter) are commonly called the “pre-Socratic philosophers.” But Socrates’ prominence is more than an accident of the calendar, for Socrates represents a shift in emphasis within Greek philosophy, away from the cosmological concerns of the sixth and fifth centuries toward political and ethical matters. Nonetheless, the shift was not so dramatic as to preclude continuing attention to the major problems of pre-Socratic philosophy. We find both the new and the old in the work of Socrates’ younger friend and disciple, Plato (fig. 2.4).\\
苏格拉底于公元前 399 年去世,这件事发生在世纪之交(当然,不是按照他们的历法,而是按照我们的历法),这使得它成为希腊哲学史上一个方便的分界点。因此,公元前六世纪和五世纪苏格拉底的前辈们(即本章到目前为止我们一直在讨论的那些哲学家)通常被称为 “前苏格拉底哲学家”。但苏格拉底的突出地位不仅仅是历法上的偶然,因为苏格拉底代表了希腊哲学重点的转变,从公元前六世纪和五世纪对宇宙论的关注转向了对政治和伦理问题的关注。然而,这种转变并没有那么剧烈,以至于排除了对前苏格拉底哲学主要问题的持续关注。我们在苏格拉底的年轻朋友及门徒柏拉图(图 2.4)的作品中既能发现新的东西,也能发现旧的东西。\\

\noindent29\\
Plato (427–348/47) was born into a distinguished Athenian family, active in affairs of state; he was undoubtedly a close observer of the political events that led up to Socrates’ execution. After Socrates’ death, Plato left Athens and visited Italy and Sicily, where he seems to have come into contact with Pythagorean philosophers. In 388 Plato returned to Athens and founded a school of his own, the Academy, where young men could pursue advanced studies (see fig. 4.1). Plato’s literary output appears to have consisted almost entirely of dialogues, the majority of which have survived. We will find it necessary to be highly selective in our examination of Plato’s philosophy; let us begin with his quest for the underlying reality.\\
柏拉图(公元前 427 - 348/47 年)出生于一个杰出的雅典家庭,该家族活跃于国家事务;他无疑是导致苏格拉底被处决的那些政治事件的密切观察者。苏格拉底死后,柏拉图离开雅典,游历了意大利和西西里岛,在那里他似乎与毕达哥拉斯学派的哲学家有了接触。公元前 388 年,柏拉图回到雅典并创办了自己的学校 —— 学园,在那里年轻人可以进行深造(见图 4.1)。柏拉图的文学作品似乎几乎全部由对话构成,其中大部分都留存了下来。我们会发现,在研究柏拉图的哲学时,有必要进行高度的筛选;让我们从他对潜在现实的探寻开始。\\

\noindent30\\
In a passage in one of his dialogues, the Republic, Plato reflected on the relationship between the actual tables constructed by a carpenter and the idea or definition of a table in the carpenter’s mind. The carpenter replicates the mental idea as closely as possible in each table he makes, but always imperfectly. No two manufactured tables are alike down to the smallest detail, and limitations in the material (a knot here, a warped board there) ensure that none will fully measure up to the ideal.\\
在他的一部对话录《理想国》中的一段话里,柏拉图思考了木匠制作的实际桌子与木匠脑海中桌子的理念或定义之间的关系。木匠在制作每一张桌子时都尽可能地复制脑海中的理念,但总是不完美的。没有两张制作出来的桌子在最小的细节上是完全一样的,而且材料的局限性(这里有个结,那里有块翘曲的木板)确保了没有一张桌子能完全达到理想的标准。\\

\noindent31\\
Now, Plato argued, there is a divine craftsman who bears the same relationship to the cosmos as the carpenter bears to his tables. The divine craftsman (the Demiurge) constructed the cosmos according to an idea or plan, so that the cosmos and everything in it are replicas of eternal ideas or forms—but always imperfect replicas because of limitations inherent in the materials available to the Demiurge. In short, there are two realms: a realm of forms or ideas, containing the perfect form of everything; and the material realm in which these forms or ideas are imperfectly replicated.\\
现在,柏拉图认为,存在一位神圣的工匠,他与宇宙的关系就如同木匠与其桌子的关系一样。这位神圣的工匠(造物主)依据一个理念或计划构建了宇宙,以至于宇宙及其其中的一切都是永恒理念或形式的复制品 —— 但由于造物主可利用的材料中固有的局限性,这些复制品总是不完美的。简而言之,存在两个领域:一个是形式或理念的领域,包含着万物的完美形式;另一个是物质领域,在这个领域中这些形式或理念被不完美地复制。\\

\noindent32(重点)\\
Plato’s notion of two distinct realms will seem strange to many people, and we must therefore stress several points of importance. The forms are incorporeal, intangible, and insensible; they have always existed, sharing the property of eternality with the Demiurge; and they are absolutely changeless. They include the form, the perfect idea, of everything in the material world. One does not speak of their location, since they are incorporeal and therefore not spatial. Although incorporeal and imperceptible by the senses, they objectively exist; indeed, true reality (reality in its fullness) is located only in the world of forms. The sensible, corporeal world, by contrast, is imperfect and transitory. It is less real in the sense that the corporeal object is a replica of, and therefore dependent for its existence upon, the form. The  has primary existence, its corporeal replica secondary existence.\\
柏拉图关于两个截然不同的领域的观念对许多人来说似乎很奇怪,因此我们必须强调几个重要的点。这些形式是无形的、不可触摸的、无法感知的;它们一直存在,与造物主共享永恒的属性;并且它们是绝对不变的。它们包括物质世界中万物的形式,即完美的理念。人们不会谈论它们的位置,因为它们是无形的,因此不具有空间性。尽管它们是无形的且无法被感官感知,但它们客观存在;实际上,真正的现实(完整的现实)只存在于形式的世界中。相比之下,可感知的、有形的世界是不完美且短暂的。从某种意义上说,有形物体是形式的复制品,因此其存在依赖于形式,所以它是不那么真实的。形式具有首要的存在性,其有形的复制品具有次要的存在性。\\

\noindent33\\
Plato illustrated this conception of reality in his famous “allegory of the cave,” found in book VII of the Republic. Men are imprisoned within a deep cave, chained so as to be incapable of moving their heads. Behind them is a wall, and beyond that a fire. People walk back and forth behind the wall, holding above it various objects, including statues of humans and animals; the objects cast shadows on the wall that is visible to the prisoners. The prisoners see only the shadows cast by these objects; and, having lived in the cave from childhood, they no longer recall any other reality. They do not suspect that these shadows are but imperfect images of objects that they cannot see; and consequently they mistake the shadows for the real.\\
柏拉图在其著名的 “洞穴寓言” 中阐释了这种关于现实的观念,该寓言见于《理想国》第七卷。人们被囚禁在一个深深的洞穴里,被锁链束缚着,以至于无法移动他们的头部。在他们身后有一堵墙,墙的另一边有一堆火。人们在墙后走来走去,举着各种各样的物体,包括人和动物的雕像;这些物体在囚犯们能看到的墙上投下影子。囚犯们只能看到这些物体投下的影子;而且,由于他们从小就生活在洞穴里,他们不再记得任何其他的现实。他们没有怀疑这些影子只是他们看不到的物体的不完美影像;因此,他们把影子误认为是真实的。\\

\noindent34(重点)\\
So it is with all of us, says Plato. We are souls imprisoned in bodies. The shadows of the allegory represent the world of sense experience. The soul, peering out from its prison, is able to perceive only these flickering shadows, and the ignorant claim that this is all there is to reality. However, there do exist the statues and other objects of which the shadows are feeble representations and also the humans and animals of which the statues are imperfect replicas. To gain access to these higher realities, we must escape the bondage of sense experience and climb out of the cave, until we find ourselves able, finally, to gaze on the eternal realities, thereby entering the realm of true knowledge.\\
柏拉图说,我们所有人都是如此。我们是被囚禁在肉体中的灵魂。这个寓言中的影子代表着感官经验的世界。灵魂从它的囚牢中向外窥视,只能感知到这些闪烁的影子,无知的人声称这就是现实的全部。然而,确实存在那些影子所微弱代表的雕像和其他物体,以及那些雕像所不完美复制的人和动物。为了接触到这些更高层次的现实,我们必须摆脱感官经验的束缚,爬出洞穴,直到我们最终能够凝视永恒的现实,从而进入真正知识的领域。\\

\noindent35\\
What are the implications of these views for the concerns of the pre-Socratic philosophers? First, Plato equated his forms with the underlying reality, while assigning derivative or secondary existence to the corporeal world of sensible things. Second, Plato has made room for both change and stability by assigning each to a different level of reality: the corporeal realm is the scene of imperfection and change, while the realm of forms is characterized by eternal, changeless perfection. Both change and stability are therefore genuine; each characterizes something; but changelessness belongs to the forms and thus shares their fuller reality.\\
这些观点对前苏格拉底哲学家所关注的问题有何影响呢?首先,柏拉图将他的形式等同于潜在的现实,同时将可感知事物的物质世界赋予派生的或次要的存在性。其次,柏拉图通过将变化和稳定分别分配到不同的现实层面,为两者都留出了空间:物质领域是不完美和变化的场景,而形式的领域则以永恒、不变的完美为特征。因此,变化和稳定都是真实的;每一个都表征着某些东西;但不变性属于形式,因而具有更完整的现实性。\\

\noindent36\\
Third, as we have seen, Plato addressed \textbf{epistemological} questions, placing observation and true knowledge (or understanding) in opposition. Far from leading upward to knowledge or understanding, the senses are chains that tie us down; the route to knowledge is through philosophical reflection. This is explicit in the Phaedo, where Plato maintains the uselessness of the senses for the acquisition of truth and points out that when the soul attempts to employ them it is inevitably deceived.\\
第三,如我们所见,柏拉图探讨了认识论问题,将观察与真正的知识(或理解)置于对立的位置。感官远非引领我们向上通往知识或理解,而是束缚我们的锁链;通往知识的途径是通过哲学思考。这在《斐多篇》中表述得很明确,柏拉图在其中坚持认为感官对于获取真理是无用的,并指出当灵魂试图运用感官时,它不可避免地会受到欺骗。\\

\noindent37(重点)\\
Now the short account of Plato’s epistemology frequently ends here; but there are important qualifications that it would be a serious mistake to omit. Plato did not, in fact, dismiss the senses altogether, as Parmenides had done and as the passage from the Phaedo might suggest Plato did. Sense experience, in Plato’s view, served various useful functions. First, sense experience may provide wholesome recreation. Second, observation of certain sensible objects (especially those with geometrical properties) may serve to direct the soul toward nobler objects in the realm of forms. Plato used this argument as justification for the pursuit of astronomy. Third, Plato argued (in his theory of reminiscence) that sense experience may actually stir the memory and remind the soul of forms that it knew in a prior existence, thus stimulating a process of recollection that will lead to actual knowledge of the forms.\\
现在,对柏拉图认识论的简短阐述通常到此为止;但有一些重要的限定条件,如果省略的话将是一个严重的错误。实际上,柏拉图并没有像巴门尼德那样完全摒弃感官,尽管《斐多篇》中的那段话可能会让人觉得柏拉图是这样做的。在柏拉图看来,感官经验有各种有用的功能。首先,感官经验可以提供有益的消遣。其次,对某些可感知对象(尤其是那些具有几何属性的对象)的观察可能有助于引导灵魂朝向形式领域中更高贵的对象。柏拉图用这个论点作为追求天文学的正当理由。第三,柏拉图(在他的回忆理论中)认为,感官经验实际上可能会激起记忆,让灵魂想起它在前世所知晓的形式,从而激发一个回忆的过程,这个过程将引导人们获得对形式的真正知识。\\

\noindent38\\
Finally, although Plato firmly believed that knowledge of the eternal forms (the highest, and perhaps the only true, form of knowledge) is obtainable only through the exercise of reason, the changeable realm of matter is also an acceptable object of study. Such studies serve the purpose of supplying examples of the operation of reason in the cosmos. If this is what interests us (as it sometimes did Plato), the best method of exploring it is surely to observe it. The legitimacy and utility of sense experience are clearly implied in the Republic, where Plato acknowledged that a prisoner emerging from the cave first employs his sense of sight to apprehend living creatures, the stars, and finally the most noble of visible (material) things, the sun. But if he aspires to apprehend “the essential reality,” he must proceed “through the discourse of reason unaided by any of the senses.” Both reason and sense are thus instruments worth having; which one we employ on a particular occasion will depend on the object of study.\\
最后,尽管柏拉图坚信对永恒形式的知识(最高的,也许是唯一真正的知识形式)只能通过理性的运用来获得,但可变的物质领域也是一个可接受的研究对象。这样的研究有助于提供理性在宇宙中运作的实例。如果这是我们感兴趣的(就像柏拉图有时也感兴趣的那样),那么探索它的最佳方法肯定是观察它。感官经验的合法性和实用性在《理想国》中得到了明确暗示,柏拉图在其中承认,从洞穴中出来的囚徒首先运用他的视觉去感知生物、星辰,最后是可见(物质)事物中最崇高的太阳。但如果他渴望理解 “本质的实在”,他必须 “通过不受任何感官辅助的理性论述” 来进行。因此,理性和感官都是值得拥有的工具;我们在特定场合使用哪一个将取决于研究的对象。\\

\noindent39\\
There is another way of expressing all of this, which may shed light on Plato’s achievement. When Plato assigned reality to the forms, he was, in fact, identifying reality with the properties that classes of things have in common. The bearer of true reality is not (for example) this dog with the droopy left ear or that one with the menacing bark, but the idealized form of a dog shared (imperfectly, to be sure) by every individual dog—those characteristics by virtue of which we are able to classify all of them as dogs. Therefore, to gain true knowledge, we must set aside all characteristics peculiar to things as individuals and seek the shared characteristics that define them into classes. Now stated in this modest fashion, Plato’s view has a distinctly modern ring. Idealization is a prominent feature of a great deal of modern science; we develop models or laws that overlook the incidental in favor of the essential. However, Plato went beyond this, maintaining not merely that true reality is to be found in the common properties of classes of things, but also that this common property (the idea or form) has objective, independent, and indeed prior existence.\\
还有另一种表达这一切的方式,这可能会阐明柏拉图的成就。当柏拉图将现实赋予形式时,实际上他是将现实与事物类别所共有的属性等同起来。真正现实的承载者不是(例如)这只左耳耷拉着的狗或那只叫声吓人的狗,而是每只狗(当然是不完美地)所共有的理想化的狗的形式 —— 那些使我们能够将它们都归类为狗的特征。因此,为了获得真正的知识,我们必须把事物作为个体所特有的所有特征放在一边,去寻求将它们定义为类别的共同特征。以这种适度的方式表述,柏拉图的观点具有明显的现代意味。理想化是许多现代科学的一个显著特征;我们发展模型或定律,忽略偶然因素而关注本质。然而,柏拉图超越了这一点,他不仅坚持认为真正的现实存在于事物类别的共同属性中,而且还认为这种共同属性(理念或形式)具有客观、独立且实际上先于存在的性质。\\
\newpage

\end{document}